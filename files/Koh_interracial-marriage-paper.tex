\input{preamble}
\renewcommand{\labelitemi}{$\bullet$}
\renewcommand{\labelitemii}{$\bullet$}
\setlength\parindent{20pt}
\newcommand*{\mymatrix}[1]{%
  \ensuremath{\left[\begin{tabular}{@{}l@{}} #1 \end{tabular}\right]}}
\DeclareCaptionFormat{cont}{#1 (cont.)#2#3\par}
\newenvironment{fignote}{\begin{quote}\footnotesize}{\end{quote}}
\renewcommand{\footnoterule}{\vfill\kern -3pt \hrule width 0.4\columnwidth \kern 2.6pt}
%\usepackage{times}
\usepackage{fourier}
\usepackage[T1]{fontenc}


\begin{document}
	
	
\pagenumbering{gobble}% Remove page numbers (and reset to 1)
\clearpage

\title{\Large Evolution and Welfare Implications of Racial Sorting in the US Marriage Market}
\author{Yu Kyung Koh\footnote{Department of Economics, McGill University. Email: \href{mailto:yu.koh@mcgill.ca}{yu.koh@mcgill.ca}. I am indebted to Pierre-Andr\'e Chiappori, Bernard Salani\'e, and Brendan O'Flaherty for their invaluable guidance and support throughout this project. I also thank So Yoon Ahn, Amanda Awadey, Sandra Black, Martsella Davitaya, Paul Koh, Rodrigo Soares, Krzysztof Zaremba, and the participants at Columbia Applied Microeconomic Theory colloquium, Southern Economic Association Annual Meeting, and various seminars for their helpful comments. All errors are my own.}}
\date{\today \\ \vspace{5mm} \href{https://yukyungkoh.github.io/files/Koh_interracial-marriage-paper.pdf}{Click here for the latest version} }

\maketitle
\onehalfspacing

\begin{abstract}
\noindent  Interracial marriage has steadily increased in the US, indicating positive progress toward social integration. Nevertheless, this progress has been uneven across different social groups, with notable gender disparities among Blacks and Asians. The gender disparity in interracial marriage is concerning as it can diminish the marriage opportunities of the gender that intermarries less than its opposite-gender counterpart. This paper analyzes how the gender asymmetries in marital gains associated with interracial marriage have evolved and shaped the distribution of welfare in the marriage market. Using a structural model of marriage market equilibrium, I first show that marital gains from interracial marriage have improved only for some pairs, revealing substantial gender and education gaps. I then show that these uneven benefits across interracial marriage types have improved marriage probabilities for some groups (e.g. college-educated Black men) while limiting marriage prospects for others (e.g. Black women). I find that both (1) sex ratio imbalance and (2) gender gaps in marital gains from interracial marriage contributed to the gender gaps in marital prospects, but differently for different racial groups. Lastly, I show that removing the gender gaps in racial preferences in marriage, particularly in the direction of stronger racial integration, would improve marital prospects for all.  
\end{abstract}

\clearpage 
\pagenumbering{arabic}



% ================================== I. INTRO ====================================== %
\pagebreak
\section{Introduction} \label{sec:intro}



Since 1967 Loving v. Virginia decision removed all legal barriers, the overall interracial marriage rate has steadily increased in the US, indicating progress toward social integration. Nevertheless, this progress has been uneven across different social groups, with notable gender disparities among Blacks and Asians.\footnote{For example, 24\% of married Black men aged 35-44 were interracially married in 2019. In contrast, 11\% of married Black women were.} A concern with these gender disparities is that they potentially diminish the marriage opportunities of the gender that engages less in interracial marriage. This is particularly concerning for Black women who already face documented challenges in finding eligible partners \textit{within} their racial group, due to the high incarceration and unemployment of Black men \citep{CharlesLuoh_2010_MaleIncarcerationMarriage, CaucuttGunerRauh_2021_BlackWhiteMarraigeGap, Liu_2020_Incarceration}. The fact that Black women intermarry (so-called "marry out") much less than Black men can \textit{further} deplete the pool of marriageable men for Black women, as discussed in \cite{Banks_book_2012}. Despite this concern, we know little about how these gender asymmetries have evolved and shaped the distribution of welfare in the marriage market. 

To comprehend the interracial marriage patterns and marital prospects for each social group, it is important to examine the structure of \textit{marital gains} and its evolution. Depending on one's own and spouse's characteristics, interracial marriages may yield varying benefits, commonly referred to as marital gains.\footnote{Marital gains, also known as marital surplus, can be conceptualized as the additional well-being derived from each type of marriage compared to staying single.  The exact nature of marital gains is complicated. As discussed in \cite{Chiappori_2020_AnnuRevEconMarriage}, marital gains encompass (i) economic gains from marriage and (ii) nonmonetary aspects, including love and social stigma, the latter of which can be important in the case of interracial marriage.} When there exist gender disparities in marital gains, they can lead to gender gaps in marriage probabilities. To better understand this, suppose that the stigma attached to interracial marriage decreases more for Black men than for Black women. This would result in higher marital gains for interracial marriages involving Black men, expanding their marriage pool. Conversely, Black women would face \textit{double-sided} challenges that worsen their marriage probabilities. On the one hand, Black women would face more difficulties marrying a \textit{Black} male partner, due to heightened competition from non-Black women. On the other hand, Black women would also have difficulties marrying a \textit{non-Black} male partner, due to the high stigma. The essential point is that the disparities in marital gains play a vital role in shaping marital prospects for each gender and racial group.\footnote{Stigma is just one example that can affect marital gains, which can also be shaped by economic considerations or other preference factors. Regardless of the underlying factors, the overall structure of marital gains determines who marries whom and who remains single.}

% Hence, Black women's marital prospects become worse due to the gender disparities in marital gains

In this paper, I investigate how marital gains from interracial marriage have changed over time and shaped the marital prospects of each social group in the US. Specifically, I address the following questions: First, how have marital gains evolved for different interracial unions depending on one's own and spouse's characteristics? Second, how have these \textit{differing} gains across interracial marriage types shaped each social group's marital prospects in equilibrium? Third, which specific changes in the marriage market related to interracial marriage contributed most to the gender gaps in marital welfare?  Fourth, what would happen if we remove the gender gaps in marital gains from interracial marriage? I answer these questions using a structural model of marriage market equilibrium. I show how differing benefits across interracial marriage types have led to uneven marital prospects across social groups, and how removing the gender gaps in marital gains --  particularly in the direction of stronger racial integration -- would improve marital prospects for all. 
%I show that differing benefits across interracial marriage types have improved the marital prospects of some groups but not for others. I show that removing the gender gap in marital gains for interracial marriage, especially in the direction of higher social integration, would improve marital welfare for all. 

%I show that differing marital gains across interracial marriage types have significantly improved marriage probabilities for some groups (e.g. college-educated Black men) while limiting marriage prospects for others (e.g. Black women). I show that removing the gender gaps in racial preferences in marriage and moving towards complete racial integration in the marriage market would improve marital welfare for all. 

%

These questions have received little attention in the literature. Existing literature on interracial marriage has mostly focused on measuring the preferences for marrying same-race partners, which is often called the preferences for ``racial endogamy" \citep{QianLichter_2011_InterracialMarriage, CiscatoWeber_2020_EvolvingMaritalPreferences, AnderbergVickery_2021_EthnicMaritalSortingUK}. These measures are informative in understanding, for example, whether Black people have increasingly weaker preferences to marry within race. However, these preferences for racial endogamy are not gender-specific. Furthermore, these measures say nothing about how each social group's marital prospects are affected by the changing structure of marital preferences. Therefore, the gender disparities in interracial marriage remain to be understood, along with their impacts on shaping marital prospects for each social group. This paper seeks to fill this gap in the literature. 


To answer the research questions, I must address a well-known challenge in estimating marital gains.\footnote{For an overview of challenges in estimating marital gains, see \cite{ChiapporiSalanie_2016_EconometricsMatching} and \cite{Chiappori_2020_AnnuRevEconMarriage}.} It may be tempting to think that marital gains simply correspond to observed marriage patterns. However, population supplies also independently affect marriage patterns. Specifically, sex ratio imbalance in population, resulting from education, immigration, or incarceration, can create a gender asymmetry in interracial marriage. For example, more women are now getting college degrees than men do, across all races, as highlighted in \cite{GoldinKatzKuziemko_2006_GenderGapCollege}. This has led to a disproportionately larger marriage pool for college-educated Black men than for college-educated Black women, which could have contributed to their gender gap in interracial marriage rates. Hence, it is important to distinguish the separate roles of changing marital gains and changing population, as they have different implications for interracial marriage patterns.  The structural matching model provides a framework to disentangle these factors. 

I proceed with the analysis in the following steps. To address the first research question, I estimate marital gains from 1980 to 2019 using a transferable utility matching model in the spirit of \cite{ChooSiow_2006_WhoMarriesWhomandWhy}. Although the estimated marital surplus by itself does not unveil its determinants, it is still informative for the following reasons. First, it helps us understand which types of interracial marriages have become more attractive and easier to form than others, shedding light on the progress of social integration across various social groups. Second, irrespective of the specific reasons behind why certain marital surpluses are higher than others (e.g. social stigma, economic consideration, friction, etc.), the overall structure of marital surplus itself determines who marries whom and who remains single. Therefore, marital surplus helps us understand why some demographic groups intermarry less than others. Using the US Census data, I estimate how marital surpluses have changed, based on the husband's and wife's race/ethnicity\footnote{I include four major races/ethnicities in the US, which are non-Hispanic White, Black, Hispanic, and Asian. Note that Hispanic is an \textit{ethnicity}, which is a social group that shares a common and distinctive culture, religion, and/or language. Although I distinguish Hispanics as a distinct group, ethnicity and race are not the same terms.} and education. 

% First, to understand the disparities in marital gains from interracial marriage and their evolution, I start by estimating marital gains from 1980 to 2019 using a transferable utility matching model in the spirit of \cite{ChooSiow_2006_WhoMarriesWhomandWhy}

 %The key idea behind this model is that marriage between each group of men and women generates a "marital surplus," which determines marital sorting independent of population supplies. \cite{ChooSiow_2006_WhoMarriesWhomandWhy} provides a framework to identify marital surplus from the observed marriage patterns.

The estimates show substantial disparities in the evolution of marital gains across interracial marriage types, indicating that social integration is stronger among some groups than others. For example, the marital surpluses for interracial marriages involving college-educated pairs have increased over time, but not for non-college-educated pairs. This suggests that interracial marriages are more attractive and easier to form among college graduates, potentially due to a larger reduction in stigma. Moreover, even conditional on education, there remains a large variation in marital surpluses depending on one's gender and race, suggesting that the benefits of interracial marriage differ widely across social groups. Notably, interracial marriages involving Black men constantly have higher marital gains than those involving Black women.  This leads to the aforementioned concern that Black women face double-sided challenges in finding same-race \textit{and} different-race partners, which could lower their marriage probability. 

Next, I proceed to understand how these disparities in marital surpluses, as well as in population composition, have impacted each group's marital prospects in equilibrium. To this end, I compare each social group's marital utility -- which, in the model, corresponds to one's likelihood of getting married -- between the actual marriage market and a counterfactual racially segregated marriage market. I call this measure ``individual gains from interracial marriage."\footnote{For clarification, individual gains defined here differ from marital gains from interracial marriage. ``Marital gain" is a \textit{marriage-level} concept, capturing the overall benefit each type of marriage generates. In contrast, ``individual gain" from interracial marriage is an \textit{individual-level} concept, capturing the degree to which access to interracial marriage has improved each individual's marriage probability.} While complete racial segregation is not desirable, comparing to this benchmark helps us understand the extent to which each social group has benefitted (or not) from access to interracial marriage, taking into account all the existing disparities in the marriage market. For example, due to the aforementioned gender disparities in marital surplus, we expect that Black men would have a higher likelihood of getting married when they can marry other races, relative to the complete racial segregation benchmark. This may not be true for Black women. My approach systematically quantifies the distributional impacts of the existing disparities related to interracial marriages. 
 

%In the matching model, each social group's expected utility from marriage market is fully summarized by that group's probability of being single. To understand how interracial marriage has improved (or not improved) one's marriage probability, I compare the expected utilities in the actual marriage market with the expected utilities in the counterfactual marriage market where interracial marriage do not happen at all. This welfare measure sheds light on the marital prospects of each group in the presence of disparities in the marital surplus For example, due to the aforementioned gender gap in marital surplus for interracial marriage, we would expect Black men's marriage probability to be higher in the actual marriage market, where they can intermarry, than in the complete racial segregation scenario. I estimate how the individual gains from interracial marriage has evolved for each gender, race, and education group over the past four decades. Distinct from the previous literature, my measures help understand the individual-level welfare implications of the disparities in interracial marriage. 

I show that uneven benefits across interracial marriage types have improved marital prospects for some groups, but not for others.  Among Black people, college-educated Black men gained the most from access to interracial marriage, and their gains have shown the most pronounced increase over time. For example, access to interracial marriage has reduced their probability of remaining unmarried by 17.5\% compared to a complete segregation benchmark in 2019. In contrast, Black women have not gained at all from interracial marriage, regardless of their education level, across all years. This implies that, despite having a larger and racially diverse pool of potential partners in the actual marriage market, Black women's marital probability is the same as the complete racial segregation scenario. Results for other racial groups also show education- and gender-based gaps in the evolution of individual gains from interracial marriage.

% To advance our understanding of which specific changes 
% Or start with the fact that there have been various changes in the US marriage market 

What specific changes in the marital surplus or the population supplies have played the biggest role in shaping these uneven individual gains from interracial marriage? To better understand this, I perform decomposition analyses using the matching model. Over the past several decades, there have been various changes in marital preferences and demographic composition in the US that have altered each social group's marital prospects. When examining the effects of these changes, it is crucial to acknowledge that \textit{any} change in the marital surplus and population supplies entails a tradeoff, as it affects some groups favorably while affecting others negatively.\footnote{To better understand this, consider a population change where the number of college-educated White women increases. All else equal, this is a favorable change for all groups of men, as they have a larger marriage pool. However, this population change can lower the marriage probabilities of all other women due to increased marriage competition.} My decomposition method accounts for the equilibrium nature of the marriage market changes and systematically summarizes the impact of each change on the distribution of marital welfare. 

%It is important to note that \textit{any} change in the marital surplus and population supplies entails a tradeoff, as it affects certain groups favorably while affecting others negatively.

%Third, I examine the specific changes within the marriage market that have driven the uneven welfare gains from interracial marriage.  While the previous analyses reveal that \textit{overall} disparities in marital surplus have resulted in unequal individual gains from interracial marriage, they do not pin down \textit{which} changes in the marital surplus played the biggest role. To advance efforts in improving marital prospects for all, it is useful to identify the particular changes that contributed the most to unequal marital prospects. Moreover, it is worth noting that demographic changes also could have shaped the uneven welfare gains from interracial marriage. For example, if demographic changes resulted in a larger number of different-race partners only for certain social groups, this could have increased interracial marriage rates for those groups, even if marital preferences did not change. Using a decomposition method based on the matching model, I estimate the effects of changes in marital surplus and population at the national level, accounting for the equilibrium nature of the effects.\footnote{It should be noted that \textit{any} change in the marriage market can affect \textit{all} participants. The decomposition method in this paper accounts for the equilibrium nature of the marriage market changes.} 
% Hence, it is useful to disentangle the roles of changing marital preferences and changing population supplies. 

%Using a decomposition method based on the matching model, I estimate the effects of changes in marital surplus and population at the national level, accounting for the equilibrium nature of the effects. It should be noted that \textit{any} change in the marriage market can affect \textit{all} participants. Because there have been so many changes in the US marriage market over the past several decades, it is tricky to concisely summarize the impact of each change on marriage patterns. The benefit of the decomposition method in this paper is that it effectively summarizes a large number of equilibrium effects of changing marital surplus and population. This, in turn, enables me to identify key changes in the marriage market that contributed the most to the uneven welfare gains from interracial marriage. The key idea for this method is to apply the implicit function theorem to a system of equilibrium matching functions. I use a fine-tuning method to link the four decades of changes in marital surplus and population supplies to the changes in welfare gains. The estimated changes in welfare gains from this method closely match the changes from the data, thereby confirming the validity of this method. 

% The benefit of this decomposition method is that it effectively summarizes a large number of equilibrium effects of changing marital surplus and population
%  his captures the general equilibrium effects of each change in the marriage market condition -- that is, marital gains and population supplies -- on the welfare gains from interracial marriage for each group of men and women. 

Results from the decomposition analyses show that both (1) gender gaps in marital gains across interracial marriage types and (2) sex ratio imbalance within the population contributed to the gender disparities in marital prospects, but differently for different racial groups. Among Black people, the disparities in marital gains played a more significant role. Specifically, I find that college-educated Black men substantially gained from the rise in the marital surplus associated with marrying college-educated White and Hispanic women. These gains are large enough to render other negative forces negligible for college-educated Black men. To elaborate, one example of an unfavorable change for Black men's marriage prospects is the rise in the marital surplus for (White man, Black woman) marriages, which intensifies marriage competition for Black men. However, this adverse impact was not large enough to cancel out the positive impacts of marital surplus changes that favored college-educated Black men. Conversely, Black women's marital prospects have not improved as much from the marital surplus changes.\footnote{To elaborate, the results show that college-educated Black women also gained from the increase in the joint surplus from marriage with college-educated White men. However, this gain is partly canceled out by the increase in marital surplus for Black men and White women marriages, among other forces that adversely impact Black women's marriage prospects. Eventually, combining all the effects coming from changing marital gains, college-educated Black women did not gain from access to interracial marriage as much as college-educated Black men did.} These findings suggest that, among Black people, the structure of marital surplus has evolved in a way that is most favorable to the most educated Black men.


In contrast, population changes -- particularly the growing sex imbalance among college graduates -- played a greater role in shaping marital prospects among White people. I find that college-educated White men derived the largest benefit from the expanded marriage pool, particularly due to the increase in the number of college-educated Asian and Hispanic women. In contrast, college-educated White women's marriage prospects were weakened by this growing sex imbalance among Asian and Hispanic college graduates. All in all, the decomposition results reveal that both gender imbalances in population supplies and marital surplus have influenced the disparities in interracial/interethnic marriage and marital prospects, but different effects are at play for different social groups.  


% For Whites, I find that the rise in the welfare gains for college-educated White men is mechanically driven by the rise in the number of college-educated Asian and Hispanic women. This shows that the increase in interracial marriage for college-educated White men is a byproduct of the population changes, rather than the consequence of the increase in the gains from interracial marriage. College-educated White women did not gain at all from the population changes, which reflects that the imbalanced sex ratio among the college graduates, generated by the reversal of the college gender gap \citep{GoldinKatzKuziemko_2006_GenderGapCollege}, played an important role in driving the gender difference in welfare gains among White college graduates. These findings reveal that gender imbalance in population supplies and marital surplus all matter for disparities in interracial marriage, but differently for different social groups. 


Finally, I consider two sets of counterfactual scenarios. The first aims to assess the impacts of removing the gender gap in marital surplus associated with interracial marriage for Black people. There are two different ways to remove such gender gaps: one is to \textit{increase} the marital surplus associated with Black women's interracial marriages to the Black men's level, and the other is to \textit{decrease} the marital surplus associated with Black men's interracial marriages to the Black women's level. Interestingly, these two approaches have different impacts on marital outcomes for each social group. Elevating the marital surplus for Black women's interracial marriages substantially enhances their marriage prospects without significantly compromising those of Black men or other racial groups. Conversely, reducing the marital surplus for Black men's interracial marriages does not notably enhance Black women's marital prospects and significantly diminishes Black men's marriage rates. Therefore, these results imply that it would be socially beneficial for everyone's marital prospects if Black women's interracial marriages became more attractive so that their interracial marriages become as easy to form as Black men's interracial marriages. 



The second counterfactual exercise is to more generally assess the impacts of complete racial integration in the marriage market -- which is equivalent to race becoming no longer relevant in marital matching. I find that progress toward complete racial integration would significantly reduce the proportion of unmarried Black men and women, as well as for all other minority groups. Furthermore, racial integration would not affect the proportion of unmarried White men and women. Overall, my counterfactual exercises imply that the endeavors to integrate currently segregated social groups in the marriage market, such as Black women, is not only desirable from a normative standpoint but also would enhance social welfare in terms of everyone's marital prospects.

This paper contributes to several strands of literature. First, I contribute to the literature examining the marital preferences concerning race. Several existing studies examine interracial marriages across groups and over time using descriptive statistics \citep{Fryer_2007_InterracialMarriagel, PewResearch_2017_InterracialMarriage}. Other studies have made further progress by disentangling the changing racial preferences in marriage and changing population supplies \citep{FuHeaton_2008_RacialEducationalHomogamy, QianLichter_2011_InterracialMarriage, CiscatoWeber_2020_EvolvingMaritalPreferences, AnderbergVickery_2021_EthnicMaritalSortingUK}. However, while these measures of racial endogamy provide insight into overall trends in racial preference in marriage, they are not gender-specific and these papers do not focus on explaining the gender disparities in interracial marriage. Some studies, including \cite{Fismanetal_2008_SpeedDating} and \cite{HitschHortacsuAriely_2010_OnlineDating}, utilize speed dating experiments and online dating data to uncover the gender gaps in preferences for partners of different races. Nonetheless, these datasets do not reveal the actual matches formed in the marriage market, and they only consider one time period, lacking an examination of how marital preferences have evolved over time. My paper contributes to this literature by investigating the evolution of the gender gaps in racial preferences in marriage and their consequences on each social group's marital welfare. 

Second, my study relates to the literature on how the costs associated with particular marriages affect marriage sorting and welfare. The notion of such costs reflects the reality that certain types of marriages can be more difficult to form than others. For example, cross-border marriages can be costly due to legal issues, long distance, and cultural differences. \cite{Ahn_2022_MatchingAcrossMarkets} studies how changes in the costs of cross-border marriages affect marriage patterns and intra-household allocation. In a similar vein, marriage between immigrants and natives can be more difficult to form. \cite{Addaetal_2022_LegalStatusCulturalDistance} studies how granting legal access to migrants, which alters the attractiveness of cross-cultural marriages, affects marital sorting. My paper adds to the literature by investigating the costs associated with the formation of different types of interracial marriage. Specifically, this paper empirically demonstrates that the structure of heterogenous costs across interracial marriages -- as reflected in differing marital gains across interracial marriage types -- enhances marital prospects for some groups while disadvantaging others.


Lastly, I add to the literature on the causes of the diverging patterns in marriage. As reviewed in \cite{LundbergPollakStearns_2016_FamilyInequality}, marriage rates in the US have declined faster for high school graduates than college graduates and for Blacks than Whites. Most of the existing studies have examined the causes \textit{within} each race, such as the rising incarceration of Black men \citep{CharlesLuoh_2010_MaleIncarcerationMarriage, Liu_2020_Incarceration, CaucuttGunerRauh_2021_BlackWhiteMarraigeGap} and the decline in employment prospects for low-skilled male workers who are heavily represented in minorities \citep{AutorDornHanson_2019_FallingMarriageMarketValue}. However, marriage \textit{across} races have rarely been examined, although interracial marriage has become more prevalent and important. I add to this literature by studying how gender gaps in interracial marriage shape the disparities in each social group's marriage probability.


The rest of the paper is organized as follows. Section \ref{sec:trends} documents motivating trends that highlight difficulties in interpreting the interracial marriage patterns. Section \ref{sec:data} describes the data and the sample selection for model estimation. Section \ref{sec:model} presents the matching model and explains the estimation of the marital surplus and expected utilities. Section \ref{sec:individualgains} explains the method to measure the individual gains from interracial marriage and presents the results. Section \ref{sec:decomposition} presents the decomposition method and the results. Section \ref{sec:counterfactual} performs counterfactual simulations. Section \ref{sec:conclusion} concludes. 
%
%
%
%This paper contributes to several strands of literature. First, I contribute to the literature on the evolution of interracial marriage in the US. Existing studies have investigated the evolution of interracial marriage by developing measures of racial endogamy\footnote{The term ``racial endogamy" refers to the tendency of people to choose mates who are in the same racial group.}  \citep{FuHeaton_2008_RacialEducationalHomogamy, QianLichter_2011_InterracialMarriage, CiscatoWeber_2020_EvolvingMaritalPreferences}. While these measures provide insight into overall trends in racial preference in marriage, they are not gender-specific measures, thereby unable to explain the gender disparities in interracial marriage. Moreover, several existing literature examines interracial marriage trends across groups using descriptive statistics and provides qualitative discussions \citep{Fryer_2007_InterracialMarriagel, PewResearch_2017_InterracialMarriage}. My paper contributes to this literature by quantitatively investigating the drivers of the gender gaps in interracial marriage and the consequences of these gender gaps on the marital prospects of each social group. 
%
%%This term is not party-specific; even the racial assortative mating shows the declining trend, this measure does not specify which party (e.g. Black men, Black women, White men, or White women) is driving the decline.
%
%This paper also contributes to the literature on equilibrium marriage sorting that uses a frictionless transferable utility (TU) framework. This paper is one of a few studies that investigate racial sorting using this framework. Since \cite{ChooSiow_2006_WhoMarriesWhomandWhy} provided a benchmark model for empirically implementing the frictionless TU matching model with unobserved characteristics, this framework has been used and extended to study marriage sorting on various dimensions, including education \citep{CSW_2018_MaritalCollegePremium, CCDM_2020_AssortativeMatchingUS},  personality traits \citep{DupuyGalichon_2014_PersonalityTraitsMarriageMarket}, and income \citep{Chiapporietl_2022_MatchingIncome}. Related to my paper, there has been a growing literature on interracial/interethnic marriages, including \cite{CiscatoWeber_2020_EvolvingMaritalPreferences}, who study the changes in multi-dimensional sorting in the US; \cite{AnderbergVickery_2021_EthnicMaritalSortingUK}, who study the role of group density on racial sorting in the UK; and \cite{Addaetal_2022_LegalStatusCulturalDistance}, who study the role of legal status and cultural distance on intermarriage in Italy. However, none of these papers examines the disparities in marital surplus from interracial marriage and their distributional consequences on marital welfare. My paper is the first to estimate and investigate the \textit{individual-level} welfare gain from interracial marriage for each demographic group. 
%
%Lastly, I add to the literature on the causes of the diverging patterns in marriage. As reviewed in \cite{LundbergPollakStearns_2016_FamilyInequality}, marriage rates in the US have declined faster for high school graduates than college graduates and for Blacks than Whites. Most of the existing studies have examined the causes \textit{within} each race, such as the rising incarceration of Black men \citep{CharlesLuoh_2010_MaleIncarcerationMarriage, Liu_2020_Incarceration, CaucuttGunerRauh_2021_BlackWhiteMarraigeGap} and the decline in employment prospects for low-skilled male workers \citep{AutorDornHanson_2019_FallingMarriageMarketValue}. I add to this literature by studying how uneven costs (or benefits) of interracial marriage across groups shape the probability of singlehood for different demographic groups through marriage market equilibrium. 
%
%
%
%The rest of the paper is organized as follows. Section \ref{sec:trends} documents motivating trends that highlight difficulties in interpreting the interracial marriage patterns. Section \ref{sec:data} describes the data and the sample selection for model estimation. Section \ref{sec:model} presents the matching model and explains the estimation of the marital surplus and expected utilities. Section \ref{sec:individualgains} explains the method to measure the welfare gains from marital racial desegregation and presents the results. Section \ref{sec:decomposition} presents the decomposition method and the results. Section \ref{sec:integration} performs counterfactual simulations for complete racial integration. Section \ref{sec:conclusion} concludes. 



% ================================== II. MOTIVATING TRENDS ====================================== %
\section{Interracial Marriage Trends} \label{sec:trends}




 In this section, I provide the descriptive interracial marriage patterns and relevant demographic changes in the US, which motivate the analyses in this paper. Interracial marriage rates have not evolved in the same way across gender and education groups. Figure \ref{fig:outmarriedblacks} shows that Black men with four-year college degrees are subtantially more likely to marry out than their female counterparts and their high school graduate male counterparts. Notably, even high school graduate Black men are more likely to marry out than college-educated Black women. Similarly, Figure \ref{fig:outmarriedwhites} shows that college-educated White men marry out more than their lower-educated male counterparts and their female counterparts, although this gender difference is less stark than the case for Black people. Furthermore, Asians have also experienced a persistent gender gap in interracial marriage, as shown in Appendix Figure \ref{fig:intrateha}.

The gender gaps in interracial marriage are particularly concerning as they can potentially diminish the marriage opportunities of some groups. For example, the marriage pattern suggests that Black men's interracial marriages are easier to form than Black women's. This gender disparity may disproportionately deplete the marriage pool for Black women, who already face documented challenges in finding partners within race due to the high incarceration and unemployment rates of young Black men \citep{CharlesLuoh_2010_MaleIncarcerationMarriage, Mechoulan_2011_ExternalEffectsBlackMaleIncarceration, Liu_2020_Incarceration, CaucuttGunerRauh_2021_BlackWhiteMarraigeGap}. Therefore, it is important to understand the magnitude of consequences of these gender disparities on each social group's marital prospects.  

What drives these gender disparities? As emphasized throughout the literature on marriage markets \citep{ChiapporiSalanie_2016_EconometricsMatching, Schwartz_2013_AssortativeMating}, marriage rates capture both the gains from marriage and the population distribution, and it is not straightforward to distinguish the two from marriage rates alone. While it is natural to think that the disparities in interracial marriage rates reflect the varying marital gains across interracial marriage types (potentially due to varying stigma or other preference factors), population supplies also can shape the marriage patterns. For instance, if there is a growing sex imbalance in the population across racial groups, it might elevate interracial marriage rates for one gender by broadening the marriage options while reducing the marriage probabilities for the other gender.

%Several changes in the US population may have affected interracial marriage rates, independent from the changes in the preferences or social stigma attached to interracial marriage. First, the US population has become more racially diverse due to the rise in Hispanic and Asian immigrants. Hispanic population has increased fourfold and Asian population has increased over sixfold since 1980 \citep{PewResearch_Hispanic, PewResearch_Asian}. Therefore, the rise in interracial marriage could be a consequence of the rising Hispanic and Asian populations.  

\begin{figure}[p] \caption{Interracial Marriage Rates Among Blacks and Whites} \label{fig:intratebw}

 \begin{subfigure}[b]{0.4\textwidth}
           \centering
         \includegraphics[width=\textwidth]{../Output/CombinedACSCensus_1960-2019/Descriptive/InterracialMarriage_amongmarried_MenWomen_r2_clean_age35-44.pdf}
         \caption{Blacks} \label{fig:outmarriedblacks}
     \end{subfigure}

        \vskip\floatsep
         \begin{subfigure}[b]{0.4\textwidth}
           \centering
         \includegraphics[width=\textwidth]{../Output/CombinedACSCensus_1960-2019/Descriptive/InterracialMarriage_amongmarried_MenWomen_r1_clean_age35-44.pdf}
         \caption{Whites}  \label{fig:outmarriedwhites}
     \end{subfigure}
    \begin{fignote}
	\underline{Note:} This figure shows the proportion of those who married out of their race among married individuals of the specified group aged 35-44 in each survey year. ``HSG" refers to high school graduation or the equivalent GED. ``CG" refers to the four-year college degree or above. Data sources for this figure are: 1980 5\% sample Census, 1990 5\% sample Census, 2000 5\% sample Census, 2010 5\% sample American Community Survey (2006-2010 5 year pooled sample), 2019 5\% sample American Community Survey (2005-2019 5 year pooled sample).  Survey weight is applied. 
\end{fignote}
	
\end{figure}

An evident example of the increasing sex ratio imbalance in the US is observed among college graduates. There has been a reversal of the gender gap in college education: while more men used to earn college degrees than women in the past, the opposite is true now. While this is a well-documented trend \citep{GoldinKatzKuziemko_2006_GenderGapCollege, ChiapporiIyigunWeiss_2009_SchoolingMarriageMarket, ChuanZhang_2022_GenderGapCollege}. I additionally show in Figure \ref{fig:sexratiocg} that this reversal in the gender gap is true for \textit{all races}. This implies that there are now a larger number of potential college-educated partners, across all races, for men than for women among college graduates. Therefore, the gender differences in interracial marriage shown in Figure \ref{fig:intratebw} could be the consequence of the reversal of the gender gap in higher education, rather than the consequence of the gender differences in the marital gains associated with interracial marriage. It is also possible that different effects are at play for different social groups. Without a clear framework, it is challenging to disentangle the effects of changing marital gains associated with interracial marriage and the effects of changing demographics. 




\begin{figure}[H] \caption{Female-to-Male Sex Ratio, Among 4-Year College Graduates, Age 35-44} \label{fig:sexratiocg}
	 \includegraphics[scale=0.7]{../Output/CombinedACSCensus_1960-2019/Descriptive/SexRatioCG_byrace_1980vs2019.pdf}
\begin{fignote}
	\underline{Note:} This figure shows sex ratio (female-to-male) among college graduates aged 35-44 for each race in 1980 and in 2019, respectively. Data sources for this figure are: 1980 5\% sample Census microdata and 2019 5\% sample American Community Survey (2015-2019 5-year pooled sample). Survey weight is applied. 
\end{fignote}
\end{figure}


%What can these trends tell about changing values of interracial marriage and their subsequent effects on individual welfare? As emphasized throughout the previous literature in social sciences, the trends in raw marriage rates are difficult to interpret in the presence of the changes in marginal distributions of the types of potential partners. The interracial marriage rate can mechanically change due to population changes even when the costs of interracial marriage do not change. In order to better understand how interracial marriages can be affected by the underlying population changes, I plot the evolution of number of men and women in each race and education group in Figure \ref{fig:popchange}.
%
%
%Figure \ref{fig:popchange} shows several notable trends in the distribution of race and education. First, there has been a large increase in the number of Hispanic and Asian populations. For Hispanics, one of the most notable facts is that high school dropouts account for the largest proportion, which is in contrast with the other racial groups that observe sharp decreases in the number of high school dropouts over time. This trend partly may reflect an increase in immigration of low-skilled Hispanics. On the other hand, there has been a sharp increase in college-educated Asian men and women. 
% 
% The second notable trend in population is the evolving gender gap in college education. While there have been more college-educated men than college-educated women in the past, the trend has reversed and there are now more college-educated women than men in terms of numbers. This is a well-documented trend in the literature \citep{GoldinKatzKuziemko_2006_GenderGapCollege, ChiapporiIyigunWeiss_2009_SchoolingMarriageMarket, ChiapporiSalanie_2016_EconometricsMatching}. What Figure \ref{fig:popchange} additionally shows is that this gender gap in college education is true for \textit{all} four races.
%
%The above-mentioned changes in marginal distribution of race and education for men and women create challenges when interpreting interracial marriage rates presented in Figure \ref{fig:interracial-raw}. For example, because there have been sharp increases in Hispanic and Asian populations over time, interracial marriage may mechanically increase as there are more racially diverse potential partners in the market. It is not clear how to distinguish the mechanical effect of the changing population from the structural effect of the changing value of interracial marriages, which motivates the need for a structural model. Moreover, it is difficult to understand how much individual benefits from marital racial integration just by looking at the marriage rates. It requires a clear framework to define the notion of ``individual welfare gains" from marital racial integration, which also motivates the need for a structural model. 

To better understand these disparities in interracial marriage patterns and their welfare implication, I proceed by building a structural matching model in Section \ref{sec:model}.  This model will allow me (i) to estimate how the marital gains from interracial marriage have changed over time depending on one's own and spouse's characteristics, (ii) how the disparities in marital surplus have shaped each group's marital prospects, and (iii) disentangle the effects of changing population and changing marital gains associated with interracial marriage. 




% ================================== DATA  ====================================== %
\section{Data} \label{sec:data}


I begin by describing the data used for estimating the marriage matching model for each year spanning from 1980 to 2019. I use the US Decennial Census for years 1980, 1990, and 2000, and I use the 5-Year American Community Survey sample for the years 2010 and 2019, all of which are extracted from IPUMS \citep{IPUMS_Census}.\footnote{All data are 5-in-100 national sample of the population for the corresponding year.} The reason I start from year 1980 is because the census question on the Hispanic origin was added in 1980.\footnote{In fact, as discussed in \cite{oflaherty_book_2015}, ``Hispanic" only gained meaning around 1970 in the US.} For the previous years, the Hispanic origin of each respondent is imputed by the IPUMS based on several criteria including one's and family's birthplace, surname, and family relationship, among others. However, it is problematic to use the imputed Hispanic variable, because interracial/interethnic marriages involving Hispanics are not well-identified.\footnote{ Specifically, the occurrences of marriages between non-Hispanic whites and Hispanics are recorded to be zero in 1960 and 1970. This is because spouse's race is one of the criteria to impute the Hispanic origin of individuals for 1960 and 1970 Decennial Censuses.} 

   I impose the following sample restriction for estimation. For \textit{each} survey year, I first select the sample of currently married couples where wife is aged 35-44 and husband is aged 37-46. The lower bound of this age range is selected to exclude the ages that are too young so that people can marry in the future. The upper bound of the age range is selected to keep the age distribution across different calendar years comparable.\footnote{The choice of age range 35-44 is common in the marriage literature (e.g. \cite{CCDM_2020_AssortativeMatchingUS}, \cite{Bertrandetal_2021_SocialNorms}).} Two years of age gap between husband's age group and wife's age group reflects the fact that men tend to marry younger women, and the most common spousal age differences in the data are 1 and 2 years.\footnote{Note that this age restriction rules out couples who have spouses outside the specified age range. For example, age 35 women married to age 33 men are excluded from the sample. This age restriction is necessary to properly estimate the marital surplus and perform counterfactual analyses, but it may be problematic as it arbitrarily rules out certain age pairs.} I focus on heterosexual married couples, because same-sex marriage had not been legalized nationwide until 2015. 
   
   
   To the married couple sample, I add the sample of never-married single men and women who are in the same corresponding age groups. I do not include divorced people in the single sample to abstract away from the issues of selection into divorce. Institutionalized individuals are excluded from the estimation sample, as they are unlikely to be participating in the marriage market. Never-married singles in the estimation sample include those living with unmarried partners.\footnote{From 1990 Census and onwards, ``unmarried partner" living with the head of household can be identified.} As shown in the Table \ref{apptab:cohabit}, cohabitation among the sample of singles has increased from 11.3\% in 1980 to 24.6\% in 2019. I later perform sensitivity checks in Appendix \ref{appsec:excludecohabit} and confirm that excluding cohabiting singles do not affect the main results.  

%{\color{orange} \textbf{(YKK: Excluding cohabiting unmarrieds do not change the results much)}} 

I now describe the definition of ``type'' of men and women that is used for the estimation. I consider 4 races/ethnicities in my estimation, which are Non-Hispanic Whites, Black/African Americans, Hispanics, and Asians\footnote{``Asian" include Chinese, Japanese, and other Asians or pacific islanders.}, which is denoted as  $\mathcal{R} \equiv \{ White, Black,$ $Hispanic, Asian \}$. I exclude other races, including mixed races because their sample size is too small; Appendix Table \ref{apptab:otherrace} shows that other races, which include Native Americans, Alaska Indians, and other races, are less than 1\% of the population of interest each year; and people who reported to be mixed race, an option available from 2000 onward, make up less than 3\% for each available year. For education, I consider four levels of educational attainment: $\mathcal{E} \equiv \{HSD, HSG, SC, CG \}$, where $HSD$: high school dropout, $HSG$: high school graduate or GED with no college education, $SC$: less than 4 years of college education, and $CG$: 4 years of college education or more. Hence, the type I consider for the estimation is a combination of one's race and education:  $\mathcal{R} \times \mathcal{E}$, which consists of 16 different types. Specifically, $\mathcal{R} \times \mathcal{E} = \{ WhiteHSD, WhiteHSG, WhiteSC, WhiteCG, \ldots,$  $AsianHSD, AsianHSG, AsianSC, AsianCG \}$.



% ================================== III. MODEL ====================================== %
\section{Marriage Matching Model}  \label{sec:model}

In this section, I present the matching model that serves as the building block for the analyses of interracial marriage throughout this paper. Building on \cite{ChooSiow_2006_WhoMarriesWhomandWhy}, I construct a frictionless matching framework with perfectly transferable utility (TU) and random preferences. This framework allows me to (i) estimate the marital gains from any race-education matching in the marriage market, (ii) quantify each social group's marital welfare, and (iii) perform various decomposition and counterfactual analyses. Here, I explain how this framework works and discuss the implications of marital surplus within the context of interracial marriage.
% Describe what the model will be doing. (See page 16 of Calvo (2021))

%In this section, I present the matching model that serves as the building block for the analyses of interracial marriage throughout this paper. Building on \cite{ChooSiow_2006_WhoMarriesWhomandWhy}, I construct a frictionless matching framework with perfectly transferable utility (TU) and random preferences. This model allows me to (i) estimate the marital gains from any race-education matching, (ii) marital utility, (iii) . 

\subsection{The Setting}

In this setting, each man or woman has two traits that are observed by the analyst: race and education.\footnote{I only focus on these two traits as they are likely to be determined before marriage.  Other observable characteristics from the census data, such as the current wage and hours of work, are not used because they can be the outcomes that are endogenously determined by marriage.} Each man $i$ belongs to a type $I = (R_i, E_i) \in \mathcal{M} \equiv  \mathcal{R} \times \mathcal{E}$, where $\mathcal{R}$ and $\mathcal{E}$ denote the type spaces for race and for education, respectively. Similarly, each woman $j$ belongs to a type $J = (R_j, E_j) \in \mathcal{F} \equiv \mathcal{R} \times \mathcal{E}$. In addition, each individual has other traits that are unobservable to the analyst but are observable to all men and women.\footnote{The unobservable heterogeneity allows for richer matching patterns, which is otherwise not possible with a deterministic matching model. As discussed throughout the matching literature \citep{ChiapporiSalanie_2016_EconometricsMatching, chiappori_book_2017, GalichonSalanie_2022_CupidInvisibleHand}, the uni-dimensional deterministic matching model gives a stark prediction that matching is perfectly assortative, which is obviously unrealistic in the real world.} 
  
 A \textit{matching} indicates who marries whom, including the option of singlehood. I augment the type spaces for men and women to allow for singlehood: $\mathcal{\tilde{M}} := \mathcal{M} \cup \{ \emptyset \} $ and $\mathcal{\tilde{F}} := \mathcal{F} \cup \{ \emptyset \} $, where $\{\emptyset \}$ means no partner. Feasible matching must satisfy the population constraints, which simply means that the number of unmarried and married people of each type should match the total number of people in the marriage market for that type. Formally, the feasibility constraint is:  
 \begin{align}
 	n^I &= \mu^{I\emptyset} + \sum_J \mu^{IJ} \quad \forall \; I \label{eq:popconsI}  \\
 	m^J &= \mu^{\emptyset J} + \sum_I \mu^{IJ} \quad \forall \; J \label{eq:popconsJ} 
 \end{align}
where $n^I$ is the number of type $I$ men and $m^J$ the number of type $J$ women available in the marriage market.  $\mu^{IJ}$ denotes the number of $(I,J)$ marriages, $\mu^{I\emptyset}$ the number of single $I$ men, $\mu^{\emptyset J}$ the number of single $J$ women. 

 
In a perfectly transferable utility framework, a matching also indicates how marital surplus $z_{ij}$ generated by $(i,j)$ marriage is divided between the couple. The formal definition of the marital surplus is the total utility man $i$ and woman $j$ get when married together minus the sum of utilities man $i$ and woman $j$ get when remaining single. It should be noted that marital surplus encompasses both husband's and wife's gains from each marriage. In essence, marital surplus encapsulates all economic and non-economic gains derived from the corresponding marriage, \textit{relative to} singlehood.\footnote{The relative nature of the marital surplus is important, because the analyst cannot separately identify the utilities from marriage and the utilities from singlehood, from the marriage patterns alone. It is only possible to identify the relative utilities that a couple gets, \textit{relative to} singlehood, from the marriage patterns.}   Hence, the exact nature of marital surplus is complicated and can encompass various factors. In the next subsection, I will discuss the implications of marital surplus in the context of interracial marriage.


Marital surplus is expressed as a sum of two components:
   \begin{align}
  	z_{ij} = Z^{IJ} + \varepsilon_{ij} \label{eq:zij}
  \end{align}
  where $Z^{IJ}$ is a deterministic part of the surplus that depends only on \textit{observed} types of spouses, and $\varepsilon_{ij}$ is an idiosyncratic part of the surplus that reflects \textit{unobserved} heterogeneity in marital preferences. When Mr. $i$ and Ms. $j$ marry each other, the joint surplus $z_{ij}$ is divided between them. This is expressed as $z_{ij} = u_i + v_j$, where $u_i$ is the payoff for Mr. $i$ and $v_j$ is the payoff for Ms. $j$. While marital surplus is considered as given in the model, how it is divided between the couple is endogenously determined by marriage market equilibrium.

Similarly, the utility of singles is expressed as:
\begin{align*}
	z_{i\emptyset} &= Z^{I\emptyset} + \varepsilon_{i\emptyset} \\
	z_{\emptyset j} &= Z^{\emptyset J} + \varepsilon_{\emptyset j} 
\end{align*}
where $Z^{I\emptyset}$ and $Z^{\emptyset J}$ are normalized to zero. 
% This utility of singles is from Chiappori Annual Rev. 
 
 
 %I augment the type spaces for men and women to allow for singlehood: $\mathcal{\tilde{M}} := \mathcal{M} \cup \{ \emptyset \} $ and $\mathcal{\tilde{F}} := \mathcal{F} \cup \{ \emptyset \} $, where $\{\emptyset \}$ means no partner. Then, the representation of matching is:
% \begin{enumerate}
 	%\item A measure $d\mu$ on the set $ \tilde{\mathcal{M}} \times  \tilde{\mathcal{F}}$, such that the marginal of $d\mu$ over $\mathcal{M}$ for men (resp. for women) coincides with initial distribution of men (resp. women)
 	%\item A set of payoffs $\{ u_i, i \in  \mathcal{M} \}$ and $\{ v_j, j \in \mathcal{F} \}$ such that $u_i + v_j = z_{ij}$
% \end{enumerate}
%The first condition ensures the feasible matching. The second condition shows that the joint surplus from marriage $z_{ij}$, which is gains to marriage relative to gains to singlehood, is \textit{additively} split into the individual surpluses of the spouses. The additivity of individual surpluses arises from the property of perfectly transferable utilities \citep{ChiapporiSalanie_2016_EconometricsMatching}. 


 A matching equilibrium is achieved when (i) no Mr. $i$ or Ms. $j$ who is currently married would rather be single and (ii) no Mr. $i$ or Ms. $j$ who are not currently married together would both rather be married together than remain in their current situation. This equilibrium condition results from stability, and the stable matching is generally unique  \citep{GaleShapley_1962_StabilityofMarriage, ShapleyShubik_1971_AssignmentGameI}. 


%  The formal representation of stable matching is: 
%\begin{align*}
%	u_i &\geq 0 \quad \text{ for any } i \in \mathcal{M}, \\
%	 v_j &\geq 0 \quad \text{ for any } j \in \mathcal{F}, \\
%	  u_i + v_j &\geq z_{ij} \quad \text{ for any } (i,j) \in \mathcal{M} \times \mathcal{F}
%\end{align*}

 
 \subsection{Identification of Marital Surplus and Expected Utility}
 
 
 One of the main goals of this paper is to understand the disparities in marital gains across interracial marriage types and their evolution. Hence, I need to recover the marital surplus (Equation (\ref{eq:zij})) from the observed marriage patterns. As discussed in \cite{ChooSiow_2006_WhoMarriesWhomandWhy} and \cite{ChiapporiSalanie_2016_EconometricsMatching}, it is not possible to identify the marital surplus without further imposing a structure on the idiosyncratic terms. This is because the analyst cannot observe how people match based on traits that are not observed in the data, such as personalities or hobbies. Following \cite{ChooSiow_2006_WhoMarriesWhomandWhy}, I impose the separability assumption, which restricts matching on unobserved traits:
 
 \vspace{2mm}
 \noindent\textbf{Assumption 1 (Separability).} \textit{The joint surplus from a marriage between a type $I$ man and a type $J$ woman is of the form}
 \begin{align}
 	z_{ij} = Z^{IJ} + \alpha^J_i + \beta^I_j \:.
 \end{align}
 
 
 It is worthwhile to discuss what this assumption allows and does not allow. It allows for matching on unobservable traits, but in a restrictive way -- which is conditional on the observed types of both spouses. For example, $\alpha^J_i$ reflects that a marriage between Mr. $i$ $\in I$ and Ms. $j$ $\in J$ may occur because Mr. $i$ has unobservable traits (e.g. a certain hobby) that type $J$ women value when choosing a partner. Moreover,  $\alpha^J_i$ can also reflect that Mr. $i$ has idiosyncratic preferences for type $J$ women. Similar implications hold for $\beta^I_j$. However, the separability assumption does not allow the matching on unobserved traits of both spouses.
 
The separability assumption leads to the following property, which is crucial for identification: 

 \vspace{2mm}
 \noindent\textbf{Proposition 1 \citep{ChooSiow_2006_WhoMarriesWhomandWhy, CSW_2018_MaritalCollegePremium}.} \textit{For any stable matching, there exists values $U^{IJ}$ and $V^{IJ}$ satisfying the following property:}
\begin{itemize}
	\item \textit{Each man $i$ will match with a woman of type $J$ that maximizes $U^{IJ} + \alpha^J_i$ over $\tilde{\mathcal{F}}$}
	\vspace{-1ex}
	\item \textit{Each woman $j$ will match with a man of type $I$ that maximizes $V^{IJ} + \beta^I_j$ over $\tilde{\mathcal{M}}$.}
	\vspace{-1ex}
	\item \textit{$U^{I\emptyset}$ and $V^{\emptyset J}$ are normalized to be zero.} 
	\vspace{-1ex}
	\item \textit{$U^{IJ} + V^{IJ} = Z^{IJ}$ if $(I,J)$ match exists.}
\end{itemize}
 \textit{Proof.} See \cite{CSW_2018_MaritalCollegePremium}.


\vspace{2mm} 
 $U^{IJ}$ (resp. $V^{IJ}$) can be interpreted as the husband's (resp. wife's) portion of the deterministic part of the joint marital surplus that is shared between the spouses.  An important consequence of Proposition 1 is that the separability assumption simplifies the two-sided matching problem by turning it into a series of discrete choice problem. Husband's share of the surplus, which is $U^{IJ}$, can be obtained from a man $i$'s problem of choosing a partner type (or choosing not to marry) that maximizes his utility -- i.e. a maximization of $U^{IJ} + \alpha^J_i$ over $\tilde{\mathcal{F}}$. Given $U^{IJ}$, wife's share of the surplus is written as $V^{IJ}=Z^{IJ} -U^{IJ}$. This can be similarly obtained through a woman $j$'s problem of choosing a partner type (or choosing not to marry) that maximizes her utility, \textit{taking into account} the surplus the husband takes from each type of marriage -- i.e. a maximization of $Z^{IJ} - U^{IJ} + \beta^I_j$ over $\tilde{\mathcal{M}}$ given all $U^{IJ}$. Then, we can identify the marital surplus $Z^{IJ}$, which is simply a summation of $U^{IJ}$ and $V^{IJ}$. 

 \vspace{4mm}
 \noindent \textbf{Marital Surplus:} Marital surplus can be identified using Proposition 1 and an assumption on the distribution of unobserved preferences. Following a common practice in the literature, I assume that the unobserved heterogeneities $\alpha^J_i$ and $\beta^I_j$ are distributed as standard type-I extreme values.\footnote{\cite{GalichonSalanie_2022_CupidInvisibleHand} show that any distributions for the random terms can be used to identify the marital surplus, as long as these distributions are known ex ante.} Then, solving the model in a standard way \citep{ChooSiow_2006_WhoMarriesWhomandWhy}, I get the following formula for marital surplus $Z^{IJ}$ for husband's type $I$ and wife's type $J$:
\begin{align} 
Z^{IJ} = ln \Big(\frac{(\mu^{IJ})^2}{\mu^{I\emptyset}\mu^{\emptyset J}}\Big) \label{eq:ZIJ}
\end{align}

 Because $Z^{IJ}$ is a function of the number of marrieds and singles, marital surplus can be recovered from the observed matching patterns. Note that, unlike raw marriage rate, the above measure of marital surplus controls for the effects of demographic composition, by scaling the proportion of $I, J$ marriages by the geometric average of the proportion of unmarrieds of those racial groups. 
 
 
 \vspace{3mm}
  \noindent \textbf{Discussion of Marital Surplus:} Before highlighting why marital surplus is useful in understanding interracial marriage patterns, I discuss its limitations. First, we cannot know which party drives the value of the marital surplus. For example, if the marriage between a Black man and a White woman has a high value of marital surplus, we cannot distinguish whether it is because Black men value marriage with White women more or it is because White women value marriage with Black men more. Second, without further specification, marital surplus does not tell us the specific reasons behind its value. It encompasses any economic, social, or other benefits associated with $(I,J)$ marriage. For instance, a high marital surplus for certain interracial marriages may be due to low social stigma, economic consideration, or other preference factors. Furthermore, because the model does not consider friction in finding a partner, the marital surpluses that differ across marriage types may reflect varying levels of difficulty in finding a partner across social groups.

Despite these limitations of marital surplus, it is informative for the following reasons. First, it helps us understand which types of interracial marriages have become more attractive and easier to form than others, shedding light on the progress of social integration across various social groups. Second, although we cannot uncover the specific reasons behind the marital surpluses, the overall structure of marital surplus itself is important as it determines who marries whom and who remains single. Therefore, the structure of marital surplus sheds light on why some demographic groups intermarry less than others and why some demographic groups remain more unmarried.

\vspace{3mm}
\noindent\textbf{Expected utilities:} Another important concept, which will be used throughout this paper, is each social group's expected utilities from the marriage market. It is crucial to note that while the marital surplus is a \textit{couple-level} gain, the expected utility is an \textit{individual-level} welfare from the marriage market. I use the notion of expected utilities in the next sections to understand who benefitted from access to interracial marriage and by how much. 

Each social group's expected utility can also be easily identified and estimated in this framework. As shown in \cite{ChooSiow_2006_WhoMarriesWhomandWhy}, the expected utility from the marriage market for male type $I$ is the following:  
\begin{align}
	\bar{u}^I = E \Big[ \underset{J}{max} (U^{IJ} + \alpha^{J}_i) \Big] = ln \Big( \sum_j exp(U^{IJ}) + 1 \Big) = - ln(Pr(single \: | \: I)) \label{eq:EU}
\end{align}
The above equation shows that the expected utility of type $I$ men can be fully expressed by their probability of being single, which is a well-established property of assuming Gumbel distributed idiosyncratic terms in a discrete choice framework. In other words, high expected utility indicates a high likelihood of marriage for a social group. A similar result applies to the female type $J$, and I denote $\bar{v}^J$ the expected utility of women of type $J$. 

One remark is that the expected utility that each group gets from the marriage market can be interpreted as \textit{price} in the marriage market. $\bar{u}^I$ is the price that a woman has to pay to marry type $I$ men; after paying the price, she keeps what is left of the joint surplus from marrying type $I$ men. Similarly, $\bar{v}^J$ is the price that a man has to pay to marry type $J$ women. Like the usual prices in any type of market, these expected utilities play an important role that equate demand and supply for each type of partners in the marriage market. 

\subsection{The System of Equilibrium Matching Functions} \label{sec:systemfunctions}

Lastly, I highlight that the matching model yields a system of equilibrium matching functions. This system enables us to link population distribution and marital surplus to equilibrium matching patterns. In other words, it allows us to explore how the changes in population or marital surplus affect who marries whom and who remains single. These functions are useful as they allow counterfactual simulations and comparative statics, which will be conducted in Section \ref{sec:individualgains} and Section \ref{sec:decomposition}.

To obtain this system of matching functions, I begin by re-arranging the marital surplus formula (Equation (\ref{eq:ZIJ})) as the following:
 \begin{align}
	\mu^{IJ}_t = exp \Big(\frac{Z^{IJ}}{2} \Big)\sqrt{\mu^{I\emptyset} \mu^{\emptyset J}} \label{eq:muIJ2}
\end{align} 
Next, I plug the above expression into the feasibility constraints (Equations (\ref{eq:popconsI}) and (\ref{eq:popconsJ})) to obtain:
\begin{align}
	n^I &= \mu^{I\emptyset} + \sum_J exp \Big(\frac{Z^{IJ}}{2} \Big)\sqrt{\mu^{I\emptyset} \mu^{\emptyset J}} \quad \forall \; I \label{eq:nI} \\ 
	m^J &= \mu^{\emptyset J} + \sum_I exp \Big(\frac{Z^{IJ}}{2} \Big)\sqrt{\mu^{I\emptyset} \mu^{\emptyset J}} \quad \forall \; J \label{eq:mJ}
\end{align}

Let $K$ be the total number of types for $I$ and $J$, respectively. Then, Equations (\ref{eq:nI}) and (\ref{eq:mJ}) define a system of $2K$ matching equations with $2K$ unknowns, which are the number of single men of each type  ($\mu^{I\emptyset}$) and the number of single women of each type ($\mu^{\emptyset J}$) for all $I, J$. 

 The counterfactual simulations can be performed using the system of Equations (\ref{eq:muIJ2}), (\ref{eq:nI}), and (\ref{eq:mJ}) with any counterfactual population distribution and/or marital surplus matrix. These simulations derive counterfactual marriage patterns under any different structure of population and marital surplus. Moreover, another interesting but unexplored feature of this system of matching functions is that it can be used to quantify the effects of changing population and changing marital surplus on marital sorting. I demonstrate how this can be done in Section \ref{sec:decomposition}.	


\subsection{Estimated Marital Surplus} \label{sec:ZIJestimates}



 How have marital gains evolved for different interracial unions depending on one's own and spouse's characteristics? In this section, I present descriptive statistics of the estimated marital surplus $\hat{\mathbf{Z}}_t$ using the data described in Section \ref{sec:data}. I document how the evolution of marital gains differs across interracial marriage types, which indicates that some types of interracial marriage have become more attractive and easier to form than others. %To understand this, I estimate the marital surplus matrix $\textbf{Z}_t$ for each year $t$ using the empirical marriage patterns in the data.I document the evolutions of the marital gains across of interracial marriage across race and education groups. 

 \begin{figure}[H] \caption{Estimated Marital Surplus Matrix $\hat{\mathbf{Z}}_t$, 1980 vs 2019}      \label{fig:ZIJmain}
 %\ContinuedFloat
  %  \captionsetup{list=off,format=cont}
  \begin{subfigure}[b]{0.47\textwidth}
           \centering
         \includegraphics[width=\textwidth]{../Output/CombinedACSCensus_1960-2019/Matching/ZIJ_age37_4educrace_year1980_notext.pdf}
         \caption{1980} \label{fig:ZIJ1980}
     \end{subfigure}
      \hfill
     \begin{subfigure}[b]{0.47\textwidth}  
          \centering 
        \includegraphics[width=\textwidth]{../Output/CombinedACSCensus_1960-2019/Matching/ZIJ_age37_4educrace_year2019_notext.pdf}
         \caption{2019}  \label{fig:ZIJ2019}
        \end{subfigure}
      
        \begin{fignote} 
\underline{Note:} This figure shows a heatmap for estimated marital surplus $\hat{Z}^{IJ}_t$ for the survey year 1980 (Panel (a)) and the survey year 2019 (Panel (b)), respectively. $I$ refers to husband's type (Row) and $J$ refers to wife's type (Column). $HSD$: high school dropout, $HSG$: high school graduate with no college education, $SC$: less than 4 years of college education, and $CG$: 4 years of college education or more.
\end{fignote}
\end{figure}



\noindent \textbf{Overall changes in marital surplus:} Figure \ref{fig:ZIJmain} shows the heatmaps of the marital surplus for the survey years 1980 and 2019.\footnote{Specific values of the marital surplus are presented in Appendix Figure \ref{appfig:Z1980} for the year 1980 and Appendix Figure \ref{appfig:Z2019} for the ear 2019.} Panel (a) confirms that in 1980 the US marriage market was largely segregated by race: for all races, the same-race marriages exhibit highest values of marital surplus. Panel (b) shows several notable patterns for the year 2019. First, compared to 1980, the values of marital surplus have generally gone down for most marriages in 2019, especially for the marriages involving lower-educated people. This reflects a well-known retreat from marriage in the US \citep{LundbergPollakStearns_2016_FamilyInequality}. Second,  in 2019, the marriage market is still largely segregated by race; the same-race marriages still exhibit highest values of marital surplus across all races. Third, within each block of interracial marriages, the values of marital surplus are highest for college graduates. 


\vspace{3mm}


\noindent \textbf{Further investigation of specific changes in marital surplus:} To further understand how the marital surpluses from interracial marriage have changed, I investigate selected marital surpluses and their corresponding changes over the 1980-2019 period. Table \ref{tab:selectZIJwhitespouse} reports selected $\hat{Z}^{IJ}_t$ for marriages involving a White spouse, and Table \ref{tab:selectZIJblackspouse} reports selected $\hat{Z}^{IJ}_t$ for marriages involving a Black spouse. When describing the changes in marital surplus below, I focus on the sign of the changes rather than on the magnitude of the changes. This is because marital surplus is a \textit{non-linear} function of quantities of marriages and singles, as shown in Equation (\ref{eq:ZIJ}), which makes it difficult to directly compare the levels of the changes in $\hat{Z}^{IJ}_t$ across marital surpluses with differing starting values. 

% (TO-DO) Good to add footnote about why nonlinearity makes it difficult to compare the values. 

% To better understand how the marital surpluses from interracial marriage have changed compared to same-race marriage for different race and education groups

\begin{table}[H] \caption{Selected Marital Surplus Involving \textbf{White Spouse}} \label{tab:selectZIJwhitespouse}
\centering
\scalebox{0.83}{\begin{tabular}{llccc| lccc}
  \toprule
 & \multicolumn{4}{l}{\textbf{Panel A: Marital surplus for CG couple }}	& \multicolumn{4}{l}{\textbf{Panel B: Marital surplus for HSG couple }}	\\
&  & 1980 & 2019 & $\Delta^{2019-1980}$ &    & 1980 & 2019 & $\Delta^{2019-1980}$ \\ 
  \midrule
White-White & $Z^{WhiteCG, WhiteCG}$ & 3.01 & 2.13 & \textbf{-0.88} & $Z^{WhiteHSG, WhiteHSG}$ & 4.74 & -0.38 & \textbf{-5.12} \\ \midrule
White-Black & $Z^{WhiteCG, BlackCG}$ & -9.58 & -7.48 & \textbf{+2.10} & $Z^{WhiteHSG, BlackHSG}$ & -11.47 & -10.84 & \textbf{+0.63} \\ 
&  $Z^{BlackCG, WhiteCG}$ & -7.08 & -5.55 & \textbf{+1.53} & $Z^{BlackHSG, WhiteHSG}$ & -7.21 & -7.67 & \textbf{-0.45} \\ \midrule
White-Hispanic & $Z^{WhiteCG, HispCG}$ & -3.01 & -2.61 & \textbf{+0.40} & $Z^{WhiteHSG, HispHSG}$ & -2.46 & -6.64 & \textbf{-4.17} \\ 
 & $Z^{HispCG, WhiteCG}$ & -3.59 & -2.92 & \textbf{+0.67} & $Z^{HispHSG, WhiteHSG}$ & -2.79 & -6.24 & \textbf{-3.44} \\ \midrule
White-Asian &$Z^{WhiteCG, AsianCG}$ & -4.50 & -2.59 & \textbf{+1.91} & $Z^{WhiteHSG, AsianHSG}$ & -2.99 & -8.43 & \textbf{-5.44} \\ 
& $Z^{AsianCG, WhiteCG}$ & -5.07 & -4.07 & \textbf{+1.00} & $Z^{AsianHSG, WhiteHSG}$ & -5.00 & -9.06 & \textbf{-4.06} \\ 
   \bottomrule
\end{tabular}}
\begin{center}
%\vspace{-7pt}
\begin{minipage}{17cm}
\begin{spacing}{0.9}
{\footnotesize{\underline{Notes:} This table reports selected marital surplus for marriages involving at least one White spouse. Panel A reports marital surplus for marriages where both spouses are college graduates. Panel B reports marital surplus for marriages where both spouses are high school graduates. For $Z^{IJ}$, $I$ refers to husband's type and $J$ refers to wife's type. $\Delta^{2019-1980}$ denotes the change in corresponding marital surplus from 1980 to 2019. CG refers to 4-year college graduates, and HSG refers to high school graduates or equivalent GED.}}
\end{spacing}
\end{minipage}
\end{center}		
\end{table}

\vspace{-5ex}



Table \ref{tab:selectZIJwhitespouse} reveals several implications about the evolution of the gains to interracial marriage. First, it has become more attractive and easier to marry across racial lines for college graduates, but no evidence is shown for high school graduates.  For example, Panel A shows that the marital surpluses of all interracial marriages involving a college-educated White spouse have increased over the past four decades.\footnote{Notably, same-race marriages between college-educated White men and women has experienced decreasing marital surplus, which reflects the overall declining value of marriages as noted by the literature \cite{LundbergPollakStearns_2016_FamilyInequality}} In contrast, Panel B shows that the marital surpluses have gone down for most interracial and same-race marriages involving a White spouse with high school degree only, and the magnitude of the decline is substantial for most marriages. These differing trends by education in the marital surplus are consistent with the arguments in social sciences that education is becoming increasingly more important than race in marriage \citep{Kalmijn_1991_StatusHomogamy, Schwartz_2013_AssortativeMating}, and that college graduates are more open to interracial marriage \citep{PewResearch_2017_InterracialMarriage}. 



%This shows that there is no evidence of a lower barrier of interracial marriage among high school graduates. 

%The declining value of interracial marriage among high school graduates may reflect both (i) the declining value of marriage in general for high school graduates and (ii) the increasing barriers of marrying across racial lines for high school graduates. 
 
Second, there are gender- and race-based gaps in the attractiveness or ease of marrying across racial lines.  Among all interracial marriages involving a White spouse, Black-White marriages exhibit the lowest marital gains. Among Black-White marriages, marriages between Black men and White women have higher joint surplus than marriages between White men and Black women, both in 1980 and in 2019. This confirms that Black-White marriages are more difficult to form than other types of interracial marriages, especially for the marriages involving Black women, as widely conjectured \citep{Fryer_2007_InterracialMarriagel, oflaherty_book_2015}. There are also gender gaps in gains to White-Asian marriages: marriagse between White men and Asian women have higher surplus than marriages between Asian men and White women. These findings are consistent with the prior evidence on racial preferences in the dating market. For example, both \cite{HitschHortacsuAriely_2010_OnlineDating} and \cite{LinLundquist_2013_MateSelectionCyberspace} show using the data from dating applications that Black women and the Asian men are the groups that are least likely to send to or receive messages from dating candidates outside their race. 
 
% One exception is the marital surplus for the marriage between White HSG men and Black HSG women. However, the value of this type of interracial marriage is the lowest among all marriages involving a White HSG spouse in both 1980 and 2019, which shows that there is a still high barrier for Black HSG women to marry White HSG men and vice versa. 

\begin{table}[H]  \caption{Selected Marital Surplus Involving \textbf{Black Spouse}} \label{tab:selectZIJblackspouse}
\centering
\scalebox{0.83}{
\begin{tabular}{llccc| lccc}
  \toprule
  &  \multicolumn{4}{l}{\textbf{Panel A: Marital surplus for CG couple}}	& \multicolumn{4}{l}{\textbf{Panel B: Marital surplus for HSG couple}}	\\
 & & 1980 & 2019 & $\Delta^{2019-1980}$ &  & 1980 & 2019 & $\Delta^{2019-1980}$ \\ 
  \midrule
Black-Black & $Z^{BlackCG, BlackCG}$ & 1.63 & -0.86 & \textbf{-2.49} & $Z^{BlackHSG, BlackHSG}$ & 1.33 & -3.40 & \textbf{-4.74} \\ \midrule
Black-White &  $Z^{BlackCG, WhiteCG}$ & -7.08 & -5.55 & \textbf{+1.53} & $Z^{BlackHSG, WhiteHSG}$ & -7.21 & -7.67 & \textbf{-0.45} \\ 
 & $Z^{WhiteCG, BlackCG}$ & -9.58 & -7.48 & \textbf{+2.10} & $Z^{WhiteHSG, BlackHSG}$ & -11.47 & -10.84 & \textbf{+0.63} \\ \midrule
Black-Hispanic  & $Z^{BlackCG, HispCG}$ & -8.65 & -6.10 & \textbf{+2.55} & $Z^{BlackHSG, HispHSG}$ & -8.30 & -8.97 & \textbf{-0.67} \\ 
 & $Z^{HispCG, BlackCG}$ & -9.99 & -9.32 & \textbf{+0.67} & $Z^{HispHSG, BlackHSG}$ & -9.45 & -11.12 & \textbf{-1.67} \\ \midrule
Black-Asian  & $Z^{BlackCG, AsianCG}$ & -9.90 & -7.14 & \textbf{+2.76} & $Z^{BlackHSG, AsianHSG}$ & -6.35 & -9.95 & \textbf{-3.60} \\ 
 & $Z^{AsianCG, BlackCG}$ & -10.17 & -10.96 & \textbf{-0.79} & $Z^{AsianHSG, BlackHSG}$ & -9.50 & -13.87 & \textbf{-4.38} \\ 
   \bottomrule
\end{tabular}}
\begin{center}
%\vspace{-7pt}
\begin{minipage}{17cm}
\begin{spacing}{0.9}
{\footnotesize{\underline{Notes:} This table reports selected marital surplus for marriages involving at least one Black spouse. Panel A reports marital surplus for marriages where both spouses are college graduates. Panel B reports marital surplus for marriages where both spouses are high school graduates. For $Z^{I,J}$, $I$ refers to husband's type and $J$ refers to wife's type. $\Delta^{2019-1980}$ denotes the change in corresponding marital surplus from 1980 to 2019. CG refers to 4-year college graduates, and HSG refers to high school graduates or equivalent GED.}}
\vspace{-6ex}
\end{spacing}
\end{minipage}
\end{center}
\end{table}



 Similar to Table \ref{tab:selectZIJwhitespouse}, Table \ref{tab:selectZIJblackspouse}  also shows that the gains from interracial marriage involving a Black spouse have only increased for the college-educated. Moreover, the gains from Black-Hispanic marriages and Black-Asian marriages are lower than the gains from Black-White marriages in both years. This shows that there are higher social or cultural barriers to marrying across races among minorities, even after accounting for their relatively small proportions in the population.  The estimates also show that interracial marriages involving Black women have a lower surplus than interracial marriages involving Black men for all cases, which again confirms that there are higher barriers for Black women to marry out of their race than for Black men. This is consistent with the anecdotal evidence that Black women experience higher social pressures than Black men to marry within-race \citep{Banks_book_2012} and that they face more discrimination in the dating market \citep{Stewart_book_2020}. 


\vspace{3mm}

% How should I move on to the next analysis? 
In this section, I have shown that some types of interracial marriage have experienced greater marital gains, indicating that racial integration has become stronger for some pairs of social groups than others. The gender disparities in marital gains across interracial marriage types, particularly among Blacks and Asians, raise concerns. For example, because Black men's interracial marriages have higher gains and hence are easier to form than Black women's interracial marriages, Black women have a disproportionately unfavorable marriage pool compared to Black men. In other words, Black women face double-sided challenges in the marriage market: one in finding a same-race partner, due to increased competition from non-Black women, \textit{and} the other in finding a different-race partner, due to low surplus associated with Black women's interracial marriages. A similar argument applies to Asian men's marital prospects relative to Asian women's. This motivates my next analysis, which is to investigate how gender asymmetries in marital gains across interracial marriage types have impacted each social group's marital prospects. 

%The estimates show substantial disparities in the evolution of marital gains across interracial marriage types, indicating that social integration is stronger among some groups than others. For example, the marital surpluses for interracial marriages involving college-educated pairs have increased over time, but not for non-college-educated pairs. This suggests that the stigma of interracial marriage is reduced only among college-educated pairs. Moreover, even conditional on education, there remains a large variation in marital surpluses depending on one's gender and race, suggesting that that the benefits of interracial marriage differ widely across social groups. Notably, interracial marriages involving Black men constantly have higher marital gains than those involving Black women.  This leads to the aforementioned concern that Black women face double-sided challenges in finding same-race \textit{and} different-race partners, which could lower their marriage probability. 

%While it is useful to understand how the structural gains from interracial marriage have changed, it is difficult to understand how these changes have affected individual welfare. For example, how did the increase in the gains from interracial marriage among college graduates shown in Table \ref{tab:selectZIJwhitespouse} affect individual welfare in the marriage market? Especially since there has been disproportionate changes in the gains from interracial marriage, not every group may have benefitted from the changes in marital preferences. In order to evaluate the welfare implications of these changes, I need to (i) construct a measure of welfare from marital desegregation for each group and (ii) estimate how these various changes in marital surplus over time have affected the welfare of each group. These will be done in the following sections. 


%It is challenging to link marital surplus to the sorting patterns and individual welfare. Above documentation reveals that the value of interracial marriage widely differ across groups, even after accounting for the population distributions. It is unclear how these varying values of interracial marriage and their unbalanced evolutions over time has affected the marriage prospects of each group. 

%In the following sections, I characterize the gains from marital racial desegregation and quantify how the changes in marital surplus, as well as the changes in population distribution, affected individual welfare gains from racial desegregation in the marriage market. 

%It is not straightforward to understand how much these various changes in the marital surplus affected individual welfare. he value of interracial marriage has not improved for all groups in the same way, and it is challenging to understand the gains each group has gained from racial desegregation in the marriage market.

% Start this section with a question 



% ================================== IV. INDIVIDUAL GAINS ====================================== %
\section{Individual Gains from Interracial Marriage}  \label{sec:individualgains}


How have the \textit{differing} gains across interracial marriage types, as documented in Section \ref{sec:ZIJestimates}, shaped each social group's marital prospects in equilibrium? The structure of marital surplus, as well as population composition, may favor the marital prospects of certain groups while disadvantaging others. In this section, I examine who has benefitted from the option of interracial marriage and by how much, while considering all the disparities in marital gains and population composition. I show that uneven marital gains across interracial marriage types have improved marriage probabilities for some groups while limiting marriage prospects for others.

 To this end, I compare each social group's marital utility between the actual marriage market and a counterfactual racially segregated marriage market. Note that, in the model, the marital utility for each social group is fully summarized by their probability of remaining unmarried, or conversely, their probability of getting married (Equation \ref{eq:EU}). While complete racial segregation is undoubtedly un undesirable scenario, having this benchmark and comparing it to the current marriage market helps us understand how much each group is benefitting from access to interracial marriage, considering all the disparities in the marital gains. Essentially, what my comparison captures is how much more probable it is for each social group to get married in the actual marriage market, where interracial marriage is accessible to everyone, in contrast to the completely racially segregated marriage market. I call this measure ``individual gains from interracial marriage."

To better understand this welfare measure, consider an example of Black men and Black women. As documented in Section \ref{sec:ZIJestimates}, Black men's interracial marriages constantly have higher marital gains than Black women's interracial marriages. This means that Black men's interracial marriages are easier to form, which creates a disproportionately larger marriage pool for Black men than for Black women. Additionally,  due to an imbalanced sex ratio documented in Section \ref{sec:trends}, college-educated Black men have a larger marriage pool in terms of quantities of different-race potential partners. Therefore, we expect that Black men are likely to have more chances of getting married when there is access to interracial marriage, relative to a complete racial segregation benchmark. Conversely, due to the aforementioned gender asymmetries in marital gains and population supplies, it is unclear if Black women would have higher chances of getting married in the actual marriage market, even though they have a larger and more racially diverse marriage pool than the segregation benchmark. My estimation approach allows me to understand the distributional impacts of the existing disparities in the marriage market. 

\subsection{Estimation Strategy} \label{sec:gains-estimation}

\textbf{Counterfactual simulation for complete segregation:} I describe the steps to compute the counterfactual equilibrium marriage patterns for the counterfactual scenario of complete racial segregation. 
\begin{itemize}
	\item \textbf{Step 1:} For each survey year $t$, I take the marital surplus matrix $\hat{\mathbf{Z}}_t$ that is estimated in Section \ref{sec:ZIJestimates}.
	
	\item \textbf{Step 2:} For each $\hat{\mathbf{Z}}_t$, I replace $\hat{Z}^{IJ}_t$ by $-\infty$ for all $(I,J)$ that correspond to interracial marriage $(R_i \neq R_j)$. This guarantees that interracial marriages do not happen in the segregated marriage market. I keep the marital surpluses for all same-race marriage at their corresponding values in $\hat{\mathbf{Z}}_t$. I denote $\hat{\mathbf{Z}}^{Segregated}_t$ the resulting counterfactual marital surplus matrix for complete racial segregation. 

	\item \textbf{Step 3:} I compute the counterfactual marriage patterns for each survey year $t$, using $\hat{\mathbf{Z}}^{Segregated}_t$ and the actual population vectors $\mathbf{n}_t$ and $\mathbf{m}_t$,. This is done by applying the Iterative Projection Fitting Procedure (IPFP) on the system of matching functions represented by Equations (\ref{eq:nI}) and (\ref{eq:mJ}) in Section \ref{sec:systemfunctions}.\footnote{\cite{GalichonSalanie_2022_CupidInvisibleHand} explain that IPFP is an efficient and fast way to solve for the stable matching. This algorithm solves the system of equations defined by Equations (\ref{eq:nI}) and (\ref{eq:mJ}) iteratively, starting from the vector of arbitrary guesses $\mu^{I\empty}_{(0)}$ and $\mu^{\empty J}_{(0)}$. The intuition behind this algorithm is that the average utilities ($\bar{u}^I$ and $\bar{v}^J$) of each type of men and women act as \textit{prices} in the marriage market that equate demand and supply of partners. Hence, the algorithm adjusts the prices alternatively on each side of the market until it reaches the stable matching.}
\end{itemize}

 Through the above procedures, I obtain the counterfactual equilibrium quantities of single men and women of each type for each survey year. One remark is that complete racial segregation can be represented by \textit{any} marital surplus matrix where the entries for interracial marriages have infinitely negative values. However, because the objective is to only capture the changes in sorting patterns due to changes in the marital gains associated with interracial marriage, I choose the values of same-race marriages in $\hat{\mathbf{Z}}^{Segregated}_t$ to be same as their values in $\hat{\mathbf{Z}}_t$. 

\vspace{4mm}

\noindent \textbf{Individual gains from (access to) interracial marriage:} Using the counterfactual sorting patterns, I define and estimate ``individual gains from interracial marriage." This measure captures the excess utility -- which corresponds to excess marriage probability -- each group receives in the actual marriage market over what they would get in a completely racially segregated marriage market. 
% Although I do not directly quantify the costs of interracial marriage that may differ across groups, I implicitly account for the heterogeneous costs by comparing (i) the observed state with varying costs across individuals and (ii) the counterfactual state with the infinitely high costs of interracial marriage that are same for everyone. Hence, the welfare gains capture the differing evolution in the marital surplus for interracial marriage that are shown in Section \ref{sec:ZIJestimates}. 

 Formally, the individual gains from interracial marriage  for each man type $I$ are defined as the following:
\begin{align}
	Gain^{I}_{m,t} = \bar{u}^{I, Actual}_t - \bar{u}^{I, Segregated}_t  \label{eq:gIm}
\end{align}
where $\bar{u}^{I, Actual}_t$ is the expected utility for each race-education-gender group in the actual marriage market and $\bar{u}^{I, Segregated}_t$ is the corresponding expected utility in the racially segregated marriage market in year $t$. Analogously, for each woman type $J$, the individual gains from interracial marriage are:
\begin{align}
	Gain^{J}_{f,t} = \bar{v}^{J, Actual}_t - \bar{v}^{J, Segregated}_t\label{eq:gJf}
\end{align}

\noindent As mentioned above, the welfare gain essentially can be understood as the difference in the prevalence of singlehood between the actual world and the counterfactual world with complete racial segregation. For example, if fewer type $I$ men remain single in the actual world than in the completely racially segregated world, it means that access to interracial marriage has increased the average welfare of type $I$ men through an increase in marriage probabilities.

%As shown in Equation (\ref{eq:EU}), the type-specific expected utilities are fully summarized by the probabilities of singlehood for each type under the assumption of Gumbel distributed stochastic terms. 

For ease of interpretation, I rescale the welfare gains to represent the percentage change in the single rate that would occur when the marriage market is completely segregated. To explain, note that $Gain^I_{m,t}$ can be re-written as:
\begin{align*}
	Gain^{I}_{m,t} &= ln \Big( Pr( Single \: |  \: I, t, Segregated ) \Big) - ln \Big( Pr( Single  \: |  \: I, t, Actual ) \Big) \\[3pt] 
	&\approx  \frac{ Pr( Single \: |  \: I, t, Segregated )  - Pr( Single  \: |  \: I, t, Actual ) }{Pr( Single  \: |  \: I, t, Actual ) }
\end{align*}
Therefore, the welfare gain multiplied by 100 can be interpreted as the percentage change in the single rate of type $I$ men that would occur if we move from the current state to complete segregation in each year $t$. 


 It should be noted that this welfare gain measure is silent about specific mechanisms that have driven the individual gains from interracial marriage. A positive individual gain \textit{only captures} the fact that the single rate of a given type is lower in the actual world than in the completely racially segregated world. It does not tell us which parts of marital surplus or the population distribution drive the gains. In Section \ref{sec:decomposition}, I investigate the specific mechanisms in the marriage market that drive the evolution of uneven individual gains from interracial marriage. 


\subsection{Results} \label{sec:gains-results}

In this section, I present the results for the evolution of individual gains from access to interracial marriage for different social groups. Although my analysis includes all four racial/ethnic groups, I focus on describing the results for Black and White people, because these two racial groups have experienced substantial increases in interracial marriage rates over the past four decades as shown in Appendix Figure  \ref{fig:intmarriage-byrace}, while Hispanics and Asians have had almost constant interracial marriage rates over time. Moreover, Hispanics and Asians are largely comprised of immigrants, and different generations of immigrants may have systematically different preferences for same-race marriage \citep{LichterCarmaltQian_2011_HispanicInterracialMarriage, furtado_2015}. It is outside the scope of this paper to investigate welfare gains for Hispanics and Asians who exhibit markedly different interracial marriage trends and immigration trends from Blacks and Whites.
% Here, describe why I am focusing more on marital gains for Whites & Blacks. 

%Figure \ref{fig:ipfp} presents the estimated welfare gains from interracial marriage over the past four decades for Blacks and Whites. To facilitate the comparison of the magnitude of the welfare gains, I use the same scale on the y-axis for all groups. I discuss the results for each group below.



\begin{figure}[p] \caption{Individual Gains from Interracial Marriage}      \label{fig:ipfp}
 \begin{subfigure}[b]{0.47\textwidth}
           \centering
         \includegraphics[width=\textwidth]{../Output/CombinedACSCensus_1960-2019/Counterfactual/InterPrem_uexpCS_age37_race4educ_black_1960to2019.pdf}
         \caption{Black Men}   \label{fig:ipfp-blackmen}
     \end{subfigure}
      \hfill
     \begin{subfigure}[b]{0.47\textwidth}  
          \centering 
        \includegraphics[width=\textwidth]{../Output/CombinedACSCensus_1960-2019/Counterfactual/InterPrem_vexpCS_age37_race4educ_black_1960to2019.pdf}
         \caption{Black Women} \label{fig:ipfp-blackwomen}
        \end{subfigure}   
        \vskip\floatsep
         \begin{subfigure}[b]{0.47\textwidth}
           \centering
         \includegraphics[width=\textwidth]{../Output/CombinedACSCensus_1960-2019/Counterfactual/InterPrem_uexpCS_age37_race4educ_white_1960to2019.pdf}
         \caption{White Men} \label{fig:ipfp-whitemen}
     \end{subfigure}
      \hfill
     \begin{subfigure}[b]{0.47\textwidth}  
          \centering 
        \includegraphics[width=\textwidth]{../Output/CombinedACSCensus_1960-2019/Counterfactual/InterPrem_vexpCS_age37_race4educ_white_1960to2019.pdf}
         \caption{White Women} \label{fig:ipfp-whitewomen}
        \end{subfigure}          
       \begin{fignote} 
\underline{Note:} These figures plot the welfare gain from interracial marriage as defined by Equation (\ref{eq:gIm}) and Equation (\ref{eq:gJf}) for each specified type of men and women. Note that welfare gain can be thought of as a change in marriage probability due to access to interracial marriage, as explained in the text. Data used to calculate the gains are: 1980-2000 Decennial Census, 2010 and 2019 5-Year ACS. I focus on age 37-46 men and age 35-44 women for each survey year. Further details on the sample restriction are described in Section \ref{sec:data}. Shade for each line refers to the 95\% confidence interval. Standard errors are calculated from the sampling variation in the data. $HSD$: high school dropout, $HSG$: high school graduate with no college education, $SC$: less than 4 years of college education, and $CG$: 4 years of college education or more.
\end{fignote}
\end{figure}



\vspace{3mm}
\noindent\textbf{Results for Black people:} Figure \ref{fig:ipfp} shows clear gender- and education-based gaps in the evolution of individual gains for Black men and women. Even as early as in 1980, Black men had on average positive welfare gains from the access to interracial marriage as shown in Figure \ref{fig:ipfp-blackmen}, although these positive gains were not significantly different from zero. Over the years, the most educated Black men experienced the highest increase in welfare gains.  In 2019, the magnitude of the welfare gains for college-educated Black men is substantial:  in the absence of access to interracial marriage, the probability of being single would be on average 17.5\% higher for the college-educated Black men. For all years, there is a clear positive relationship between Black men's education level and the individual gains from interracial marriage. 

 In contrast, Figure \ref{fig:ipfp-blackwomen} shows that Black women did not gain from interracial marriage across all years. While the welfare gain for college-educated Black women increased over time from the negative mean value in 1980, this increase is not as large as what their male counterpart has experienced.

This reveals a less-discussed aspect of the currently low marriage rates of Black women. Previous literature has focused on the explanations related to the within-race marriage market; the low marriage rate of Black women is typically attributed to the lack of marriageable Black men \citep{CharlesLuoh_2010_MaleIncarcerationMarriage, Liu_2020_Incarceration, CaucuttGunerRauh_2021_BlackWhiteMarraigeGap}. Figure \ref{fig:ipfp-blackwomen} provides an explanation pertaining to the across-race marriage market. Given the existing structure of marital surplus and population supplies, Black women do not benefit from access to interracial marriage at all, which further contributes to their low marriage rates. 

%Incarceration explains how the lack of marriageable men at the lower end of the spectrum has affected the marriage prospects of Black women. In contrast, Figure \ref{fig:ipfp-black} reveals that marital racial desegregation has taken Black men from the higher end of the spectrum, and Black women have not enjoyed any benefits from marriage options with other races as much as Black men have.


\vspace{4mm}
\noindent\textbf{Results for White people:} Figure \ref{fig:ipfp-whitemen} shows that college-educated White men experienced a larger increase in welfare gains than their less-educated counterparts: in 2019,  interracial marriage led to a reduction in their probability of being single by 8\% compared to the complete segregation scenario. In contrast, non-college-educated White men experienced a slight decrease in welfare gains over time. From 2000 onward, there is a clear positive relationship between White men's education level and the welfare gains they receive from access to interracial marriages.  

Welfare gains for White women show different patterns. While all White women did not gain at all from interracial marriage in 1980, they increasingly gained more over time. Notably, there is no clear education difference in the trends, unlike the case of White men. In 2019, access to interracial marriage reduced the probability of being single for White women, across all education levels, by  3 $\sim$ 5\% relative to the complete segregation scenario. 


\vspace{3mm}
\noindent To sum up, I have shown the gender- and education-based gaps in the evolution of individual gains from access to interracial marriages. Overall, college-educated men benefited the most from racial desegregation in recent years among Blacks and Whites. In the next section, I investigate the specific changes in the US marriage market that have driven these uneven welfare gains. 

% Next, I proceed to understand how these disparities in marital surpluses have impacted each group's marital prospects in equilibrium. To this end, I compare each social group's marital utility -- which, in the model, corresponds to one's likelihood of getting married -- between the actual marriage market and a counterfactual racially segregated marriage market. I call this measure ``individual gains from interracial marriage."\footnote{For clarification, individual gains defined here differ from marital gains from interracial marriage. ``Marital gain" is a marriage-level concept, capturing the overall benefit each type of marriage generates. In contrast, ``individual gain" from interracial marriage is an individual-level concept, capturing the degree to which access to interracial marriage has improved each individual's marriage probability.} While complete racial segregation is not desirable, comparing to this benchmark helps us understand the extent to which each social group has benefitted (or not) from access to interracial marriage, taking into account all the existing disparities in marital surplus. For example, due to the aforementioned gender disparities in marital surplus, we expect that Black men would have a higher likelihood of getting married when they can marry other races, relative to the complete racial segregation benchmark. This may not be true for Black women. My approach systematically quantifies the distributional impacts of the existing disparities of marital gains across interracial marriage types. 


%Moreover, it should be noted that it is important to distinguish Hispanics and Asians by immigration status, as different generations of immigrants may have systematically different preferences for same-race marriage \citep{LichterCarmaltQian_2011_HispanicInterracialMarriage, furtado_2015}. It is outside the scope of this paper to examine welfare gains for Hispanics and Asians who exhibit markedly different interracial marriage trends and immigration trends from Blacks and Whites. 


% ================================== V. COMPARATIVE STATICS ====================================== %
\section{Decomposition} \label{sec:decomposition}

% Start the section with a question
What drove the uneven benefits of access to interracial marriage? In this section, I examine how the changing marital surplus and changing demographic composition have shaped the uneven gains from interracial marriage. Using a decomposition method based on the matching model, I identify which changes have been most important to each social group's marital prospects. 


It is important to note that \textit{any} change in the marital surplus and population supplies entails a tradeoff, as it affects certain groups favorably while affecting others negatively. To better ilustrate this, I consider two examples of marriage market changes:  one related to marital surplus and the other to population supplies. First, consider an increase in the marital surplus associated with (Black man, White woman) marriages. This change favors Black men's marital prospects: although the number of women in the market remains constant, this change essentially expands the marriage pool for Black men, as marriages with White women have become more attractive. However, all else equal, this is an unfavorable change to Black women's and White men's marital prospects, because they would experience heightened marriage competition. Furthermore, other racial groups' marital prospects -- such as Hispanics and Asians' -- can also be affected by this marriage market change. 

Similar logic applies to the following example of a population change. Consider that the number of White women in the marriage market has increased. All else equal, this would increase Black men's gains from access to interracial marriage by expanding their marriage pool, while reducing the marital prospects of Black women. Therefore, when examining any marriage market change, we need to take into account how it would impact everyone's marital prospects. 

%Consequently, the growing gender imbalance in marital surplus or population supplies can lead to gender gaps in marital prospects. For example, when Black men's interracial marriages yield higher marital gains than Black women's interracial marriages, this can lead to disproportionately favorable marital prospects for Black men. Indeed, as documented in Section \ref{sec:ZIJestimates}, there always have been substantial gender gaps in marital surplus for Black men's and Black women's interracial marriages. Furthermore, a growing sex ratio imbalance can also contribute to a gender gap in marital prospects. As documented in Section \ref{sec:trends}, increasingly more women across all races are obtaining college degrees compared to men. This has disproportionately expanded the pool of different-race partners for Black men than for Black women with college degrees. Hence, these population changes could also have contributed to the gender gap in the individual gains from interracial marriage shown in \ref{sec:individualgains}.

 The matching model outlined in Section \ref{sec:model} enables me to account for the equilibrium nature of marriage market changes, by tracking how \textit{each} marriage market change -- which is favorable to some groups while being unfavorable to others --  can impact \textit{all} participants. The decomposition method effectively summarizes the equilibrium effects of a large number of marriage market changes that happened in the US over the past several decades. By doing so, it helps us identify the most important changes that explain the gender gaps in marital prospects for each racial group. 

% Note that for any social group, there is positive change and negative change, both from marital surplus and population. Using an example of Black men, let's explain all the forces that positively and negatively affect their marital prospects. 


%What specific changes in the marital surplus or the population supplies have played the biggest role in shaping these uneven individual gains from interracial marriage? To better understand this, I examine the effects of changes in marital surplus and population supplies. Over the past several decades, there have been various changes in marital preferences and demographic composition in the US that have altered each social group's marital prospects. When examining the effects of these changes, it is crucial to acknowledge that \textit{any} change in the marriage market can affect \textit{all} participants.\footnote{To better understand this, consider a population change where the number of college-educated White women increases. All else equal, this is a favorable change for all groups of men, as they have a larger marriage pool. However, this population change can lower the marriage probabilities of all other women due to increased marriage competition.} My decomposition method accounts for the equilibrium nature of the marriage market changes and systematically summarizes the impact of each change on the distribution of marital welfare. 

%Third, I examine the specific changes within the marriage market that have driven the uneven welfare gains from interracial marriage.  While the previous analyses reveal that \textit{overall} disparities in marital surplus have resulted in unequal individual gains from interracial marriage, they do not pin down \textit{which} changes in the marital surplus played the biggest role. To advance efforts in improving marital prospects for all, it is useful to identify the particular changes that contributed the most to unequal marital prospects. Moreover, it is worth noting that demographic changes also could have shaped the uneven welfare gains from interracial marriage. For example, if demographic changes resulted in a larger number of different-race partners only for certain social groups, this could have increased interracial marriage rates for those groups, even if marital preferences did not change. Using a decomposition method based on the matching model, I estimate the effects of changes in marital surplus and population at the national level, accounting for the equilibrium nature of the effects.\footnote{It should be noted that \textit{any} change in the marriage market can affect \textit{all} participants. The decomposition method in this paper accounts for the equilibrium nature of the marriage market changes.} 
% Hence, it is useful to disentangle the roles of changing marital preferences and changing population supplies. 

%Using a decomposition method based on the matching model, I estimate the effects of changes in marital surplus and population at the national level, accounting for the equilibrium nature of the effects. It should be noted that \textit{any} change in the marriage market can affect \textit{all} participants. Because there have been so many changes in the US marriage market over the past several decades, it is tricky to concisely summarize the impact of each change on marriage patterns. The benefit of the decomposition method in this paper is that it effectively summarizes a large number of equilibrium effects of changing marital surplus and population. This, in turn, enables me to identify key changes in the marriage market that contributed the most to the uneven welfare gains from interracial marriage. The key idea for this method is to apply the implicit function theorem to a system of equilibrium matching functions. I use a fine-tuning method to link the four decades of changes in marital surplus and population supplies to the changes in welfare gains. The estimated changes in welfare gains from this method closely match the changes from the data, thereby confirming the validity of this method. 

% The benefit of this decomposition method is that it effectively summarizes a large number of equilibrium effects of changing marital surplus and population
%  his captures the general equilibrium effects of each change in the marriage market condition -- that is, marital gains and population supplies -- on the welfare gains from interracial marriage for each group of men and women. 


\subsection{Overview of the Method} \label{sec:comparativemethod}
 In this section, I explain each step I conduct to perform decomposition analyses. The idea is to use the system of equilibrium matching functions in Section \ref{sec:model} to track the impacts of changing marital surplus and changing population. To accomplish this, I combine the implicit function theorem (IFT) and a fine-tuning method. While the application of IFT to the matching model has previously been suggested \citep{ChooSiow_2006_WhoMarriesWhomandWhy}, to the best of my knowledge, this method has not yet been implemented in the literature. I demonstrate how this can be done. 
 
\vspace{3mm}
\noindent \textbf{Implicit differentiation.}  It is worth mentioning that the equilibrium matching functions (Equations (\ref{eq:nI}) and (\ref{eq:mJ})) are interdependent on one another, specifying how any given marital surplus and population supply relates to the marriage pattern for all. Hence, applying IFT to the whole system of matching functions enables me to estimate how any small change in the marriage market affects the equilibrium sorting patterns. 

%The first step of this method is to use the implicit function theorem  on the system of equilibrium matching functions defined by Equations (\ref{eq:nI}) and (\ref{eq:mJ}).

To facilitate the application of IFT, I start by applying the following changes of variables: $\tilde{Z}^{IJ}_t = exp \Big( \frac{Z^{IJ}_t}{2} \Big)$ and $s^{I\emptyset}_t = \sqrt{\mu^{I\emptyset}_t}$ and $s^{\emptyset J}_t = \sqrt{\mu^{\emptyset J}_t}$. I use $\tilde{\boldsymbol{\theta}}_t = (\mathbf{n}_t, \mathbf{m}_t, \mathbf{\tilde{Z}}_t)$ to denote a vector of all components of marital surplus and population supplies.\footnote{Full expansion of $\tilde{\boldsymbol{\theta}}_t $ is $\tilde{\boldsymbol{\theta}}_t = \Big( n^1_t, \ldots, n^K_t, m^1_t, \ldots, m^K_t, \tilde{Z}^{11}_t, \tilde{Z}^{12}_t, \ldots, \tilde{Z}^{KK}_t \Big)$. This vector has $2K + K^2$ components; because $K=16$ in my setting, $\tilde{\boldsymbol{\theta}}_t$ has 288 components.} Then, Equations (\ref{eq:nI}) and (\ref{eq:mJ}) are re-written as:
\begin{align}
	F^I(\tilde{\boldsymbol{\theta}}_t) &= (s^{I\emptyset}_t)^2 + \sum_J \tilde{Z}^{IJ}_t s^{I\emptyset}_t s^{\emptyset J}_t - n^I_t = 0 \quad \quad \forall \; I \label{eq:nI2} \\ 
	G^J(\tilde{\boldsymbol{\theta}}_t) &= (s^{\emptyset J}_t)^2 + \sum_I \tilde{Z}^{IJ}_t s^{I\emptyset}_t s^{\emptyset J}_t - m^J_t = 0 \quad \quad \forall \; J \label{eq:mJ2}
\end{align}

%\noindent It should be noted that As the $2K$ matching equations defined by Equations (\ref{eq:nI2}) and (\ref{eq:mJ2}) are not independent, I need to apply the IFT on the \textit{whole} system in order to get partial derivatives of $s^{I\emptyset}_t$ for all $I$ and  $s^{\emptyset J}_t$ for all $J$ with respect to each of the marginal changes in the model primitives $\tilde{\boldsymbol{\theta}}_t$. 

Applying IFT on the whole system of equations (\ref{eq:nI2}) and (\ref{eq:mJ2}) leads to the following Jacobian matrix, which summarizes the impact of each marriage market change on the quantity of unmarried people in each social group. 

\begin{align}
\begin{Large}
	\begin{bmatrix}
		 \frac{\partial \mathbf{s}_t }{\partial \tilde{\boldsymbol{\theta}}_t}
	\end{bmatrix}_{(2K) \times (2K+K^2)}
	= - 
	\begin{bmatrix}
		\frac{\partial \mathbf{F}}{\partial  \mathbf{s}_t } \\ \addlinespace 
		\frac{\partial \mathbf{G}}{\partial  \mathbf{s}_t } 
	\end{bmatrix}^{-1}_{(2K)\times (2K)}
	\begin{bmatrix}
		\frac{\partial \mathbf{F}}{\partial \tilde{\boldsymbol{\theta}}_t } \\ \addlinespace 
		\frac{\partial \mathbf{G}}{\partial \tilde{\boldsymbol{\theta}}_t}
	\end{bmatrix}_{(2K) \times (2K+K^2)} 
	\end{Large} \label{eq:IFT}
\end{align}
For the brevity of notation, I use $\mathbf{s} = ( s^{1\emptyset}, \ldots, s^{K\emptyset}, s^{\emptyset 1}, \ldots, s^{\emptyset K} )$ to denote a vector of the square roots of the number of unmarried people in each social group. $\mathbf{F}$ is a vector for $F^I$ and $\mathbf{G}$ is a vector for $G^J$. The full solution for the partial derivatives is presented in Appendix \ref{appsec:decomp-method}. 

\vspace{3mm}
\noindent \textbf{Linking the Jacobian to welfare gains from interracial marriage.} Recall from Section \ref{sec:individualgains} that the individual gain from interracial marriage compares each social group's singlehood rate in the current marriage market versus that in a complete racial segregation scenario. In other words, individual gain is a function of each group's singlehood rates, as shown in Equations (\ref{eq:gIm}) and (\ref{eq:gJf}). Therefore, it is straightforward to use the Jacobian matrix (Equation (\ref{eq:IFT})) to understand how a small change in
marital surplus and population supplies affects the individual gains from interracial marriage. 

To demonstrate, consider the expected utility for type $I$ men. Recall that this expected utility takes the following form: $\bar{u}_t^I = - ln(Pr(single \: | \: I,t)) = -ln \Big( \frac{\mu_t^{I\emptyset}}{n_t^I} \Big)$ and $\mu_t^{I\emptyset} = (s_t^{I\emptyset})^2$. Then, the total differential of $\bar{u}_t^I$ is:
\begin{align}
	\underbrace{d\bar{u}_t^I}_{\substack{\text{Change in } \\ \text{marital utility}  \\ \text{of type I men}}} &= \underbrace{\frac{1}{n_t^I}dn^I}_{\substack{\text{Contribution from} \\ \text{the change in \#}  \\ \text{of type I men}}}  \underbrace{- \frac{2}{s_t^{I\emptyset}} \Big(  \frac{\partial s_t^{I\emptyset} }{\partial \tilde{\boldsymbol{\theta}}_t} d \tilde{\boldsymbol{\theta}}_t\  \Big)}_{\substack{\text{A vector of} \\ \text{contribution from}  \\ \text{each market change}}} \label{eq:duI}
\end{align}
where the partial derivative $ \frac{\partial s^{I\emptyset}_t }{\partial \tilde{\boldsymbol{\theta}}_t}$ is from the Jacobian matrix shown in Equation (\ref{eq:IFT}). 

Note that the right-hand side of Equation (\ref{eq:duI}) can be \textit{linearly} decomposed into parts that are attributed by each change in the marital surplus and population supplies, $d\tilde{\theta}^k_t$. This is the key feature that allows the decomposition of the change in each social group's gain from interracial marriage \textit{into} contributions made by each change in the population distribution and the marital surplus. This will be further explained in the next steps. 


Because the welfare gain is the \textit{difference} between each social group's expected utility in the actual world and the expected utility in the completely segregated world, this can be expressed analogously using Equation (\ref{eq:duI}).  

%Because Equation (\ref{eq:duI}) is a linear function in $d \boldsymbol{\tilde{\theta}}$, it provides a way to decompose the change in $u^I$ into the contributions from each of the changes in the model primitives $\boldsymbol{\tilde{\theta}}$. For example, the part of $d\bar{u}^I$ that is driven by the change in number of type $L$ men ($n^L$) is estimated by $-\frac{2}{s^{I\emptyset}} \Big(  \frac{\partial s^{I\emptyset} }{\partial n^L} d n^L  \Big)$.

%Moreover, I can estimate the sensitivity of $\bar{u}^I$ with respect to changes in each of the model primitives using Equation (\ref{eq:duI}). Sensitivity with respect to each model primitive $\theta$ is simply the multiplier that is attached to $d\theta$. For example, the sensitivity with respect to changes in number of type $L$ men ($n^L$) is estimated by $-\frac{2}{s^{I\emptyset}} \Big(  \frac{\partial s^{I\emptyset} }{\partial n^L} \Big)$.


\vspace{5mm}
\noindent \textbf{Fine-tuning method to link four decades of changes in $\tilde{\boldsymbol{\theta}}$.} The implicit function theorem approach  only applies to small changes in the model primitives. However, the objective is to understand the effects of four decades of changes in population and in marital surplus on welfare gains. Using the implicit function approach for such large changes may led to an incorrect decomposition. 
%Before explaining the homotopy method, I first describe the motivation for this method. A naive way of expressing the changes in $\bar{u}^I$ from year 1980 to 2019 using Equation (\ref{eq:duI}) is the following:
%\begin{align*}
%	\Delta^{2019-1980} \bar{u}^I = \frac{1}{n^I} \Delta^{2019-1980} n^I -  \frac{2}{s^{I\emptyset}} \Big(  \frac{\partial s^{I\emptyset} }{\partial \tilde{\boldsymbol{\theta}}} \Delta^{2019-1980} \tilde{\boldsymbol{\theta}}  \Big)
%\end{align*}
%where $\Delta^{2019-1980} y$ refers to change in $y$ from 1980 to 2019. However, this is problematic because the implicit function theorem and the total differentials only give good approximations for \textit{very small} changes in the model primitives. As shown in Figure \ref{fig:popchange}, US has experienced large changes in population distribution over the past four decades. Moreover, marital surplus $\mathbf{Z}$ also has experienced changes over time. Hence, it is improper to use 40 years of changes to evaluate Equation (\ref{eq:duI}). 

%A better, but still not ideal, approach is to divide the time period into smaller time periods based on available survey years. Because I use the census data with 10-year intervals, $\Delta^{2019-1980} \bar{Gains}^I_t$ can be decomposed into:
%\begin{align*}
%	\Delta^{2019-1980} \bar{Gains}^I = \Delta^{1990-1980} \bar{Gains}^I +  \Delta^{2000-1990} \bar{Gains}^I + \Delta^{2010-2000} \bar{Gains}^I + \Delta^{2019-2010} \bar{Gains}^I 
%\end{align*}
%%where 
%%\begin{align}
%%	\Delta^{t_2-t_1} \bar{u}^I = \frac{1}{n^I_{t_1}} \Delta^{t_2-t_1} n^I -  \frac{2}{s^{I\emptyset}_{t_1}} \Big( \Big[ \frac{\partial s^{I\emptyset} }{\partial \boldsymbol{\tilde{\theta}}}  \Big]_{t_1} \Delta^{t_2-t_1} \boldsymbol{\tilde{\theta}}  \Big) \label{eq:deltat}
%%\end{align}
%However, changes in model primitives over each decade may still be considered large. For example, Figure \ref{fig:popchange} shows that the 1980s experienced a sharp increase in college educated men and women. Even with the 10 year changes, the IFT and the total differentials may still result in imprecise approximations. 

To deal with this issue and better approximate the effects of changes in marital surplus and population supplies that happened over the past four decades, I implement a fine-tuning method following \cite{Judd_1998_book}. This method decomposes the large changes in the marriage market into a series of infinitesimal changes. Using this method, I evaluate the differentials for each infinitesimal change and update the approximation along the path of infinitesimal changes. I apply this method for each decade within the 1980-2019 period, based on available survey years. 

For a better understanding of this method, I illustrate using the following example. Consider the changes in the marriage market from 1980 to 1990. I denote 1980 as $\tau = 0$ and 1990 as $\tau = 1$. Then $\tilde{\boldsymbol{\theta}}_0$ (resp. $\tilde{\boldsymbol{\theta}}_1$)  is the vector of marital surplus and population supplies in 1980 (resp. in 1990). Consider the homotopy:
\begin{align*}
	\tilde{\boldsymbol{\theta}}_\tau = \tau \tilde{\boldsymbol{\theta}}_1 + (1-\tau) \tilde{\boldsymbol{\theta}}_0, \quad \tau \in [0,1]
\end{align*}
which defines a series of intermediate values of marital surplus and population supplies with interval $d\tau$ between observed values at $\tau = 0$ and $\tau = 1$. Because  the vector of marital surplus and population $\tilde{\boldsymbol{\theta}}_\tau$ is now a function of $\tau$, Equation (\ref{eq:duI}) is re-written as:
\begin{align}
	d\bar{u}^I_{\tau} &= \frac{1}{n^I_\tau}(n^I_1 - n^I_0)d\tau -  \frac{2}{s^{I\emptyset}_{\tau}} \Big(  \frac{\partial s^{I\emptyset}_{\tau} }{\partial \tilde{\boldsymbol{\theta}}_\tau}  (\tilde{\boldsymbol{\theta}}_1 - \tilde{\boldsymbol{\theta}}_0)d\tau  \Big) \label{eq:duI-homotopy}
\end{align}
%Note that $s^{I\emptyset}_\tau$, the equilibrium number of single $I$ men, is updated as $\tau$ progresses with interval $d\tau$. 

Using the above specification, I can use infinitesimal change $d\tau$ in marital surplus and population to evaluate and decompose the infinitesimal changes in $\bar{u}^I_t$ between the years 1980 and 1990.  In practice, I specify $d\tau = 0.001$ when estimating Equation (\ref{eq:duI-homotopy}) for each decade. Summing the decompositions of infinitesimal changes of the expected utility over the 1980-1990 period gives a better approximation of $\Delta \bar{u}^I_t$ than directly using the observed 10-year changes in model primitives to evaluate Equation (\ref{eq:duI}). 

Application to the four decades of changes in individual gains from access to interracial marriage is done analogously. Further details on this method are provided in Appendix \ref{appsec:decompdetails}. 


\vspace{5mm} 
\noindent \textbf{Decomposition.} The steps outlined above result in a linear decomposition of four decades of changes in individual gains, which are shown in Section \ref{sec:individualgains}, into contributions from changes in population and  marital surplus. For instance, the welfare gain from access to interracial marriage for type $I$ man between 1980 to 2019 are decomposed into the following:
	\begin{align}
	\Delta^{2019-1980} Gain^I &= (Contribution \: by \: \Delta n^1) + \ldots + (Contribution \: by \: \Delta n^K)  \nonumber \\
	&\quad + (Contribution \: by \: \Delta m^1) + \ldots + (Contribution \: by \: \Delta m^K)  \nonumber  \\ 
	&\quad + (Contribution \: by \: \Delta Z^{11}) + \ldots + (Contribution \: by \: \Delta Z^{KK}) \label{eq:decomposition}
\end{align}
As shown in Equation (\ref{eq:decomposition}), this method can summarize a large number of the contributions from various market-level changes. Therefore, this allows me to identify which changes have driven the individual gain from interracial marriage for each group of men and women over the past four decades. 



I evaluate the validity of the decomposition method by comparing the estimated changes in welfare gains from the data (using Equations (\ref{eq:gIm}) and (\ref{eq:gJf})) and the estimated welfare gains using the IFT approach. The latter is simply the total sum of contributions for each group, as represented by the right-hand side of Equation (\ref{eq:decomposition}). Table \ref{tab:comparedgains} shows that the estimates based on the method closely match those based on the data for each group of man and woman, confirming the validity of the method. 

\begin{table}[H] \caption{Data vs. IFT: 1980-2019 Changes in Welfare Gains from Interracial Marriage} \label{tab:comparedgains}
\scalebox{1}{
\begin{tabular}{lcc @{\hskip 0.28in} cc}										
\toprule										
	&	\multicolumn{2}{c}{Men}			&	\multicolumn{2}{c}{Women}			\\	\cmidrule(lr){2-3} \cmidrule(lr){4-5}
Type	&	Data	&	IFT	&	Data	&	IFT	\\	\midrule
WhiteHSD	&	 -0.728  &  -0.730 &   4.826  &     4.827	\\	
WhiteHSG	&	 -0.492  &  -0.494  &  6.018   &  6.021 	\\	
WhiteSC	&	1.076     &  1.074   &  5.727  &    5.729	\\	
WhiteCG	&	5.231   &  5.230   & 4.020   &  4.021 \\	
BlackHSD	&	 2.566   &  2.567    & 1.515   &  1.516	\\	
BlackHSG	&	1.254 &   1.255   & 2.575 &  2.576\\	
BlackSC	&	 4.114    &   4.115   &   3.366    &  3.366	\\	
BlackCG	&	 8.028    & 8.027    &  4.635   &    4.637	\\	
	\bottomrule
\end{tabular}																						
}
\begin{center}
\begin{minipage}{14cm}
\begin{spacing}{0.9}
{\footnotesize{\underline{Notes:} This table reports the changes in (i) welfare gain from access to interracial marriage of each group over the 1980-2019 period that is estimated from data and (ii) the corresponding changes estimated from the IFP method.  ``Data" column refers to the estimated change from the data, and ``IFT" column refers to estimated changes using IFT according to Equation (\ref{eq:decomposition}). $d\tau = 0.001$ is used when applying the fine-tuning method. }}
\end{spacing}
\end{minipage}
\end{center}
\end{table}

\vspace{-4ex}



\subsection{Decomposition Results}

\subsubsection{Overall effects of changing marital surplus and changing population}

Before describing the full results, I first summarize the effects of overall marital surplus changes and overall population changes on each social group's welfare gain from interracial marriage. Table \ref{tab:decompoverall} that some social groups benefitted more from marital surplus changes, while others benefitted more from population changes. For instance, for BlackCG men, combined changes in marital surplus substantially improved their gains from access to interracial marriage, surpassing the gains received by any other social group from the same changes.  For BlackCG men, population changes only played a minor (and negative) role in the overall increase in their welfare gains from interracial marriage. In contrast, population changes played a more important role in driving up the welfare gains that WhiteCG men received from access to interracial marriage. 

These results highlight the importance of disentangling the effects of changing marital surplus and changing populations, as different effects are at play for different social groups' marital prospects. The matching model is effective in disentangling these effects, a task that would otherwise be challenging due to the unobserved nature of marital surplus. In the next section, I explain the specific mechanisms behind these impacts.


\begin{table}[H] \caption{Decomposition of the 1980-2019 Changes in the Welfare Gain from Interracial Marriage} \label{tab:decompoverall}
	\scalebox{1}{\begin{tabular}{lccc}										
\toprule										
	&		&	\multicolumn{2}{c}{Total Contribution from the Changes in:}				\\	\cmidrule(lr){3-4} 
Type	&	$\Delta^{2019-1980} Gain$ 	&		 Population  	&	Marital Surplus\\  \midrule
BlackCG Men	&	8.0       &      -2.5   &   10.6 	\\	
BlackHSG Men &  1.3    &    -3.1  &   4.3  \\ \midrule 
BlackCG Women &   4.6   &    0.8  &    3.8  \\ 
BlackHSG Women &   2.6  &   1.0  & 1.6   \\ \midrule 
WhiteCG Men &     5.2   &  6.8   &   -1.6  \\
WhiteHSG Men &   -0.5    &  5.4    & -5.8  \\ \midrule 
WhiteCG Women &  4.0 & -1.2  &  5.2   \\
WhiteHSG Women &    6.0 &   -0.7  &  6.7  \\	\bottomrule
\end{tabular}}	
\begin{center}
%\vspace{-7pt}
\begin{minipage}{14cm}
\begin{spacing}{0.9}
{\footnotesize{\underline{Notes:} This table presents the decomposition of the 1980-2019 changes in the welfare gains from interracial marriage for the specified group. $\Delta^{2019-1980} Gains$ is the change in the welfare gains for the specified group over the 1980-2019 period. ``Population" Column shows the summation of all contributions by changes in population over the 1980-2019 period. ``Marital Surplus" Column shows the summation of all contributions by changes in marital surplus over the 1980-2019 period..}}
\end{spacing}
\end{minipage}
\end{center}									
\end{table}




\subsubsection{Role of gender gap in marital surplus: A case for Black men and women}

In this section, I describe how the evolving gender gap in marital surplus associated with interracial marriage has intensified the gender gap in marital prospects for Black men and women. Although my decomposition analyses encompass estimating the impacts of all marriage market changes, I focus on discussing the most important changes for Black men's and women's marital prospects.  For BlackCG men and BlackCG women, the gender gap in their gains from access to interracial marriage, as demonstrated in Section \ref{sec:gains-results}, was predominantly driven by changes in marital surplus, as indicated in Table \ref{tab:decompoverall}. To understand which marital surplus changes played the most important role, in Table \ref{tab:blackCG-decomp-top5}, I report the top three positive and negative contributions from the marital surplus changes for BlackCG men and women, respectively. 



\begin{table}[H] \caption{Decomposition: Top three contribution from changes in $\mathbf{Z}$, Black College Graduates} \label{tab:blackCG-decomp-top5}
\centering
\scalebox{0.9}{
\begin{tabular}{llccc}
 \toprule

   \multicolumn{5}{l}{A. Decomposition for BlackCG Men's $\Delta$ Welfare Gain \quad ($\Delta Gain = 8.0$) } \\[0.3cm]
 
\quad Contribution & Top (+) & 4.7 & 1.8 & 1.5 \\ 
   &  & $\uparrow$ $Z^{BlackCG,WhiteCG}$ & $\downarrow$ $Z^{BlackCG,BlackSC}$ & $\uparrow$ $Z^{BlackCG,HispCG}$  \\ 
   & Top (-) & -0.6 & -0.6 & -0.4  \\ 
   &  & $\uparrow$ $Z^{WhiteCG,BlackCG}$ & $\downarrow$ $Z^{BlackSC,BlackCG}$ & $\downarrow$ $Z^{BlackCG,AsianHSD}$  \\  \midrule 
     \multicolumn{4}{l}{B. Decomposition for BlackCG Women's $\Delta$ Welfare Gain \quad  ($\Delta Gain = 4.6$)} \\[0.3cm]
    \quad Contribution & Top (+) & 2.0 & 1.9 & 1.2  \\ 
   &  &  $\uparrow$ $Z^{WhiteCG,BlackCG}$ & $\downarrow$ $Z^{BlackCG,BlackCG}$ & $\downarrow$ $Z^{BlackSC,BlackCG}$ \\ 
   & Top (-) & -1.0 & -0.9 & -0.5 \\ 
   &  & $\downarrow$ $Z^{BlackCG,BlackSC}$ &  $\uparrow$ $Z^{BlackCG,WhiteCG}$ & $\downarrow$ $Z^{BlackCG,BlackHSG}$\\ 
   \bottomrule
\end{tabular}}
\begin{center}
%\vspace{-7pt}
\begin{minipage}{17cm}
\begin{spacing}{0.8}
{\footnotesize{\underline{Notes:} This table presents the top three positive and negative contributions from marital surplus to the 1980-2019 changes in the welfare gains from interracial marriage for Black college graduate men (Panel A) and Black college graduate women (Panel B). For marital surplus $Z^{IJ}$, $I$ refers to husband's type and $J$ refers to wife's type. Upward arrow ($\uparrow$) indicates that the corresponding marital surplus has increased over the analysis period, and downward arrow  ($\downarrow$)  indicates that it has decreased over the analysis period. }}
\end{spacing}
\vspace{-3ex}
\end{minipage}
\end{center}
\end{table}



I find that BlackCG men have benefitted the most from the increases in marital surplus associated with interracially/inter-ethnically marrying WhiteCG and HispanicCG women. Panel A in Table \ref{tab:blackCG-decomp-top5} shows that BlackCG men's gains from such changes are substantial, representing about 77.5\% of their total welfare gain from interracial marriage over the analysis period ($\Delta Gain = 8.0$). 

Notably, these positive contributions are much larger than the negative contributions from unfavorable marriage market changes for BlackCG men, rendering the latter negligible. For instance, an increase in marital surplus for (WhiteCG men, BlackCG women) marriage is un \textit{unfavorable} change for BlackCG men's marital prospects, as it would intensify marriage competition for them. However, as shown in Panel A in Table \ref{tab:blackCG-decomp-top5}, the negative contribution from this unfavorable change was negligible (-1.0), relative to the positive contributions from the increases in marital surplus for BlackCG men's interracial marriages. 

In contrast, Panel B in Table \ref{tab:blackCG-decomp-top5} shows that the marital surplus changes related to BlackCG women's interracial marriage had much smaller positive effects for BlackCG women's marriage probabilities. Specifically, although BlackCG women's marital probabilities were improved due to the rise in marital surplus associated with interracially marrying WhiteCG men, this gain is much smaller -- less than half -- of what BlackCG men gained from marrying WhiteCG women. Furthermore, the second row shows that about half of this positive gain for BlackCG women is offset by an \textit{unfavorable} change for BlackCG women's marital prospects, which is the increase in marital surplus for (BlackCG men, WhiteCG women) marriages. 

Overall, the above results imply that the gender disparities in marital surpluses associated with BlackCG men's and BlackCG women's interracial marriages played an important role in creating a gender gap in their marital prospects. For lower-educated Black people, I show in Appendix Table \ref{tab:blackHSG-decomp-top5} that none of the changes in marital surplus associated with their interracial marriages had a significant impact on their marital prospects. Therefore, my results show that the marital surplus associated with interracial marriage has evolved in a way that is most favorable to the most educated Black men among Black people. 

% Need to add a small section about lower educated BLack men and women. 

\subsubsection{Role of sex ratio imbalance: A case for White men and women}

In this section, I describe how the growing sex ratio imbalance among college graduates has improved WhiteCG men's marital prospects while diminishing WhiteCG women's. I start by showing in Table \ref{tab:decomp-whiteCG-pop} that the overall population changes in the non-White population have had varying impacts on WhiteCG men and WhiteCG women. Panel A shows that WhiteCG men benefitted from the overall increase in the non-White female population, and this positive contribution is large enough to completely offset the unfavorable marriage market change for them -- which is the increase in the non-White male population, which intensifies marriage competition for WhiteCG men.



In contrast, WhiteCG women's marital prospects were diminished by the overall changes in the non-White population. Speficially, unfavorable marriage market change for them -- which is the increase in the non-White female population -- completely offset the positive contribution from an increase in the non-White male population. 

\begin{table}[H] \caption{Contributions from population changes to welfare gains for White college graduates} \label{tab:decomp-whiteCG-pop}
\centering
\scalebox{0.9}{
\begin{tabular}{lcccc}
 \toprule
  &  \multicolumn{4}{c}{ {Population Changes 1980-2019}}  \\ \cmidrule(lr){2-5} 
 &  Non-White & Non-White & White & White  \\ 
 &  Female & Male & Female & Male \\ \midrule 
    \multicolumn{5}{l}{A. Decomposition for WhiteCG Men's $\Delta$ Welfare Gain \quad ($\Delta Gain = 5.2$) } \\[0.3cm]
 
 \quad Contribution &  34.1 & -26.3 & -5.0 & 4.0  \\   \midrule 
    \multicolumn{5}{l}{B. Decomposition for WhiteCG Women's $\Delta$ Welfare Gain \quad ($\Delta Gain = 4.0$) } \\[0.3cm]
 
 \quad Contribution &  -26.8 & 25.7 & 5.5 & -4.5  \\   \bottomrule 
\end{tabular}}	
\begin{center}
%\vspace{-7pt}
\begin{minipage}{13.5cm}
\begin{spacing}{0.9}
{\footnotesize{\underline{Notes:} This table presents the decomposition of the 1980-2019 changes in the welfare gains from access to interracial marriages for WhiteCG men and WhiteCG women. This table focuses on the contributions from the population changes. Each column shows the summation of all contributions made by populations corresponding to the label.}}
\end{spacing}
\end{minipage}
\end{center}
\end{table}

\vspace{-5ex}



\begin{figure}[H] \caption{Details on the contributions from the changes in non-White population} \label{fig:decomp-whiteCG-pop-details}

 \begin{subfigure}[b]{0.8\textwidth}
           \centering
         \includegraphics[width=\textwidth]{../Output/CombinedACSCensus_1960-2019/Counterfactual/PartialDiff_fempop_race4educ_WhiteCGmen_1960to2019.pdf}
         \caption{Decomposition for WhiteCG Men's Welfare Gains}
     \end{subfigure}
	  \hfill
	   \begin{subfigure}[b]{0.8\textwidth}
           \centering
         \includegraphics[width=\textwidth]{../Output/CombinedACSCensus_1960-2019/Counterfactual/PartialDiff_fempop_race4educ_WhiteCGwomen_1960to2019.pdf}
         \caption{Decomposition for WhiteCG Women's Welfare Gains}
     \end{subfigure}
    \begin{fignote}
	\underline{Note:} This figure presents the decomposition of the 1980-2019 changes in the welfare gains from interracial marriage for WhiteCG men and WhiteCG women. This figure focuses on the contributions of the changes in different-race populations. Each column shows the contribution made by the change in the corresponding population. 
\end{fignote}
	
\end{figure}


 I further investigate which non-White population is driving these gender differences in marital gains. As shown in Figure \ref{fig:decomp-whiteCG-pop-details}, WhiteCG men's rising welfare gains from interracial marriage are largely driven by the growing AsianCG women and HispanicCG women population. Conversely, these population changes diminished WhiteCG women's marital prospects by heightening marriage competition. Appendix Figure \ref{fig:decomp-whiteCG-pop-details2} further demonstrates that for WhiteCG men, these rises in the female college graduate population had large enough positive impacts to completely offset negative impacts from male population changes. In contrast, for WhiteCG women, the positive impacts of the rising Asian and Hispanic male population were not large enough to counteract the unfavorable female population changes in the marriage market. All in all, these results imply that the growing sex ratio imbalance among college graduates across all races, as documented in \ref{sec:trends}, played an important role in shaping the gender gap in marital prospects for White men and women. 








\subsubsection{Discussion}

Results from the decomposition analyses reveal that the underlying reasons for uneven gains from interracial marriage are complicated: some groups have gained from population changes, while other groups gained from marital surplus changes. These results highlight the importance of disentangling the effects of changing population supplies and changing marital surplus, as they have different implications.  

Another key finding is that both (i) the gender difference in marital surplus and (ii) sex ratio imbalance play a role in shaping uneven gains from access to interracial marriage. For Black men and women, I find that the evolving gender gap in the marital surplus from interracial marriages has improved college-educated Black men's marital prospects while limiting Black women's. For White men and women, the increasingly imbalanced sex ratio among college graduates across all races has substantially benefitted the marriage prospects of White men, while diminishing the marriage prospects of White women.



% ================================== VI. Complete Integration ====================================== %
\section{Counterfactual Exercises} \label{sec:counterfactual}

So far, I have shown how the distribution of population supplies and marital surplus can lead to uneven benefits from access to interracial marriage. Notably, for Black people, gender gaps in marital surplus from interracial marriage have played a crucial role in creating the gender gaps in marriage probabilities. 


What will happen if we remove the gender gaps in marital gains from interracial marriage for Black people? Furthermore, what will happen if race is no longer a factor considered in marriage matching? In this section, I simulate counterfactual scenarios using alternative marital surplus associated with interracial marriages and investigate their impacts on the marital prospects of each social group. 


\subsection{Counterfactual: Removing the gender gap in the marital surplus} \label{sec:cfgendergap}
% Look into Calvo's counterfactuals section to see how she writes 


I begin by considering the removal of the gender gaps in the marital surplus from interracial marriage. I focus on the case of Black people's interracial marriages. The goal here is to understand if such removal of the gender gap would improve Black women's marriage probabilities without harming all other groups' marriage probabilities. 

There are two different ways of removing the gender gap in marital surplus associated with Black people's interracial marriages. The first way is to elevate the marital gains associated with \textit{Black women}'s interracial marriages to match those associated with \textit{Black men}'s interracial marriages. Before formally describing this scenario, I rewrite the marital surplus $Z^{IJ}$ as $Z^{(R_i, E_i), (R_j, E_j)}$, where $R_i$ (resp. $R_j$) denotes husband's (resp. wife's) race and $E_i$ (resp. $E_j$) denotes husband's (resp. wife's) education. Let $\tilde{Z}$ denote the counterfactual marital surplus and $Z$ denote the actual marital surplus. Then, I can express this counterfactual scenario as the following: 

\vspace{2mm}
\noindent \textbf{Counterfactual 1:} \textit{Remove the gender gap in marital gains  by \textbf{increasing} the marital surplus associated with Black women's interracial marriage to be the same as those of Black men: }
\begin{align}
	\tilde{Z}^{(R_i, E_i), ({\color{red}Black_j}, E_j)} = Z^{({\color{blue}Black_i}, E_i), (R_j, E_j)} \label{eq:cf1gendergap}
\end{align}
\textit{where $R_i \neq Black_i$ and $R_j \neq Black_j$. All other $Z$s remain the same as the actual values.} 

\noindent For example, the marriage between a BlackCG woman and a WhiteCG man would now have the same surplus as the marriage between a BlackCG man and a WhiteCG woman. Counterfactual 1 is equivalent to increasing the attractiveness of interracial marriage associated with Black women, hence moving towards stronger racial integration. This can be thought of as lowering the stigma associated with Black women's interracial marriages, or as increasing marital preferences associated with those marriages. 

Alternatively, the second way of removing the gender gap is to reduce the marital gains associated with \textit{Black men}'s interracial marriages to match the same as those associated with \textit{Black women}'s interracial marriages. Specifically, 

\vspace{2mm}
\noindent \textbf{Counterfactual 2:} \textit{Remove the gender gap in marital gains  by \textbf{decreasing} the marital surplus associated with Black men's interracial marriages to be the same as those of Black women: }
\begin{align}
	\tilde{Z}^{({\color{blue}Black_i}, E_i), (R_j, E_j)} = Z^{(R_i, E_i), ({\color{red}Black_j}, E_j)} \label{eq:cf2gendergap}
\end{align}
\textit{where $R_i \neq Black_i$ and $R_j \neq Black_j$. All other $Z$s remain the same as the actual values.} 

\noindent This is equivalent to \textit{decreasing} the attractiveness of interracial marriages associated with Black men.

Using the estimated marital surplus for the year 2019, I construct the counterfactual marital surplus matrices corresponding to the first and second scenarios, respectively. Using these counterfactual marital surplus matrices, I simulate counterfactual marriage rates in each scenario and compare them to the actual marriage rates for each social group. I examine which scenario is more effective in improving marital welfare for all groups. 


\vspace{2mm}

\begin{figure}[H] \caption{Percentage Point Changes in Marriage Rates after Removing Gender Gap in $\mathbf{Z}^{interracial}$, Year 2019} \label{fig:cfgendergap-Black}

 \begin{subfigure}[b]{0.47\textwidth}
           \centering
         \includegraphics[width=\textwidth]{../Output/CombinedACSCensus_1960-2019/Counterfactual/Counterfactual_gendergap_Blackwomen_37_4educrace_year2019.pdf}
         \caption{Black Women} \label{fig:cf1-Blackwomen}
     \end{subfigure}
	  \hfill
	   \begin{subfigure}[b]{0.47\textwidth}
           \centering
         \includegraphics[width=\textwidth]{../Output/CombinedACSCensus_1960-2019/Counterfactual/Counterfactual_gendergap_Blackmen_37_4educrace_year2019.pdf}
         \caption{Black Men} \label{fig:cf2-Blackmen}
     \end{subfigure}
    \begin{fignote}
	\underline{Note:} This figure plots the percentage point differences in marriage rate between actual marriage rates and counterfactual marriage rates for each group: (Counterfactual marriage rate - Actual marriage rate). Counterfactuals consider two different scenarios of removing the gender gap in the marital surplus from interracial marriage for Black people. As described in \ref{sec:cfgendergap}, \textbf{Counterfactual 1} sets the marital surpluses of \textit{Black women}'s interracial marriages the same as those of \textit{Black men}'s interracial marriages (Equation \ref{eq:cf1gendergap}). Alternatively,  \textbf{Counterfactual 2} sets the marital surpluses of \textit{Black men}'s interracial marriages the same as those of \textit{Black women}'s interracial marriages (Equation \ref{eq:cf2gendergap}). 
\end{fignote}
	
\end{figure}


Figure \ref{fig:cfgendergap-Black} shows that the first counterfactual is substantially more effective in improving Black women's marriage probabilities without compromising Black men's. In Panel (a) of Figure \ref{fig:cfgendergap-Black}, I show that elevating the marital surplus for Black women's interracial marriages to the levels of Black men's (Counterfactual 1) improves Black women's marriage rates by 4.0 $\sim$ 5.9 p.p. In contrast,  reducing the marital surplus for Black men's interracial marriages to the levels of Black women's (Counterfactual 2) does not lead to much improvement in Black women's marriage probabilities. 


Similarly for Black men, Panel (b) of Figure \ref{fig:cfgendergap-Black} indicates that while the reductions in their marriage rates in Counterfactual 1 are small (ranging from -0.7 $\sim$ -1.2 p.p.), the adverse effects are substantial in Counterfactual 2 (ranging from -3.4 $\sim$ -5.4 p.p.). In the Appendix, I additionally show the impacts of each counterfactual on the marriage rates of other racial groups. Appendix Figures \ref{appfig:cfgendergap-nonBlack} and \ref{appfig:cfgendergap-nonBlack2} show that both counterfactual scenarios only have small negative impacts on women of other racial groups. Furthermore, the first counterfactual leads to a stronger improvement in marriage rates of men of other racial groups. 

The overall implication of the above results is that removing the gender gap in marital surplus \textit{in the direction of} making Black women's interracial marriage more attractive is not only desirable from a normative standpoint but also effective in terms of everyone's marital prospects. It would improve Black women's marriage rates without compromising the marriage rates of any other groups. Conversely, decreasing the gender gap in marital surplus by deterring Black men's interracial marriage is undesirable both from a normative standpoint and in terms of marital welfare.


\subsection{Counterfactual: Complete racial integration}

Lastly, I now turn to a scenario of racial integration\footnote{As discussed in \cite{oflaherty_book_2015}, the term ``integration" refers to a situation where groups of \textit{equals} who cooperate in mutually beneficial ways, which is different from the term ``desegregation," which simply means removing legal barriers to intergroup contact.}: If interracial marriage becomes more attractive and easier to form, how would it affect the probability of remaining single for each demographic group? I define \textit{complete racial integration} as a scenario where race is no longer a factor considered in marriage-matching. I perform counterfactual simulations to predict the effects of progress toward complete racial integration in the marriage market. 

I start by constructing a marital surplus matrix for complete racial integration. Note that this is not straightforward because, by definition, any matrix that does not depend on the race of each spouse can reflect complete racial integration. In practice, I choose a marital surplus matrix that (i) only depends on the education of both spouses and that (ii) minimizes the weighted Euclidean distance from the estimated $\hat{\mathbf{Z}}_{t}$. The marital surplus for each education pair is constructed as the weighted average of estimated $\hat{Z}^{IJ}_t$ from the data, conditional on the education levels of both spouses:  
\begin{align}
	\hat{Z}^{E_i, E_j}_t = \sum_{R_i, R_j}\widehat{Pr}( R_i, R_j | E_i, E_j, t)  \hat{Z}^{(E_i, R_i), (E_j, R_j)}_t 
\end{align}
The marital surplus matrix for complete integration, denoted by $\mathbf{\tilde{Z}}^{Integrated}_t$, is constructed by simply replacing all $\hat{Z}^{(R_i, E_i), (R_j, E_j)}_t$ in $\mathbf{\hat{Z}}_t$ with the corresponding $\hat{Z}^{E_i, E_j}_t$. Appendix Figure \ref{fig:Zcompare} visualizes the differences between the counterfactual marital surplus and the actual marital surplus. 

To construct a trajectory of progress towards the complete integration, I take the following convex combination of the marital surplus matrices:
\begin{align}
	\mathbf{\hat{Z}}^{Simulated}_t(p) = (1-p)\mathbf{\hat{Z}}^{Actual}_t + p\mathbf{\hat{Z}}^{Integrated}_t
\end{align}
where $p \in [0,1]$. This means that when $p$ is closer to 0, the counterfactual marital surplus is closer to the actual marriage market. When $p$ is closer to 1, the counterfactual marital surplus is closer to the case of complete racial integration.



%






\begin{figure}[p] \caption{Simulated Rate of Singlehood for Varying Degree of Racial Integration} \label{fig:intragrationVarying}

 \begin{subfigure}[b]{0.47\textwidth}
           \centering
         \includegraphics[width=\textwidth]{../Output/CombinedACSCensus_1960-2019/Counterfactual/CompleteIntegration_VaryingPercent_Blackmen_2019.pdf}
         \caption{Black Men} \label{fig:integrationBlackmen}
     \end{subfigure}
	  \hfill
	   \begin{subfigure}[b]{0.47\textwidth}
           \centering
         \includegraphics[width=\textwidth]{../Output/CombinedACSCensus_1960-2019/Counterfactual/CompleteIntegration_VaryingPercent_Blackwomen_2019.pdf}
         \caption{Black Women} \label{fig:integrationBlackwomen}
     \end{subfigure}
        \vskip\floatsep
         \begin{subfigure}[b]{0.47\textwidth}
           \centering
         \includegraphics[width=\textwidth]{../Output/CombinedACSCensus_1960-2019/Counterfactual/CompleteIntegration_VaryingPercent_Whitemen_2019.pdf}
         \caption{White Men}  \label{fig:integrationWhitemen}
     \end{subfigure}
     \hfill 
         \begin{subfigure}[b]{0.47\textwidth}
           \centering
         \includegraphics[width=\textwidth]{../Output/CombinedACSCensus_1960-2019/Counterfactual/CompleteIntegration_VaryingPercent_Whitewomen_2019.pdf}
         \caption{White Women}  \label{fig:integrationWhitewomen}
     \end{subfigure}
    \begin{fignote}
	\underline{Note:} This figure plots the simulated rate of singlehood at each \% of racial integration (rescaled $p$) for the specified group. ``\% Racially Integrated" describes how close the counterfactual marital surplus is to the complete integration case. Estimation is done using the 2019 Census data. I focus on age 37-46 men and age 35-44 women. Further details on the sample restriction are described in Section \ref{sec:data}.
\end{fignote}
	
\end{figure}


Figure \ref{fig:intragrationVarying} shows that, based on the 2019 data, progress towards complete racial integration would reduce the singlehood among Black men and women. 50\% of racial integration in the marriage market would reduce the single rate of Black high school graduate women by 17 p.p. and reduce the single rate of Black college-educated women by 20 p.p. Further progress toward racial integration not only closes the racial gap in marriage but also makes Blacks marry more than Whites. Appendix Figure \ref{fig:integrationVarying2} also demonstrates that other racial/ethnic minority groups would experience increased marriage rates if race/ethnicity becomes less significant in marriage-matching. In comparison, I find that racial integration would not change the rates of singlehood among Whites at all stages. 

Overall, these predictions show that racial integration would play an important role in improving the marriage prospects for Black men and women, who currently have low marriage rates, while not harming the marriage prospects for White men and women. More generally, the results show that racial minorities would substantially benefit when race is no longer relevant in marriage matching (``complete racial integration"), as it significantly expands their marriage pool. 






% ================================== VII. Conclusion ====================================== %
\section{Conclusion}  \label{sec:conclusion}

Although still largely segregated by race, the US marriage market has experienced an overall steady increase in interracial marriage. While this surely is positive and desirable progress towards social integration, the disparities in interracial marriage rates, especially the gender gap, raise a concern that some social group's marital prospects may be more limited than others. 

This paper investigates the evolution and welfare implications of the disparities in interracial marriage patterns using a structural model of marriage market equilibrium. I show that there have been wide disparities in marital gains across interracial marriage types, suggesting that the progress of social integration is stronger for some pairs than others. Moreover, I show that due to these disparities in marital gains and population composition, certain groups, such as college-educated Black men, have experienced significant improvements in marital prospects from access to interracial marriages, while others, like Black women, have not.

In terms of what specific changes in the marriage market drive these gender gaps, I find that the (i) evolving gender gap in marital surplus associated with interracial marriage and (ii) increasing sex ratio imbalance both played a role, albeit in different ways for various groups. For Black men and women, I find that the gender gap in the marital surplus from interracial marriages played a bigger role in shaping the gender gap in marital prospects. In contrast, for White men and women, the increasingly imbalanced sex ratio among college graduates has benefitted the marriage prospects of White men, while diminishing the marriage prospects of White women. Decomposition results also highlight that any given marriage market change entails a tradeoff, favoring some groups' marital prospects while diminishing other's. 


Simulation results show that reducing the gender gap in marital surplus associated with Black people's interracial marriage, particularly in the direction of stronger social integration, would enhance Black women's marriage probabilities without compromising others'. In contrast, reducing the marital surplus associated with Black men's interracial marriages to the level of Black women's is not only undesirable normatively but also ineffective in terms of everyone's marital welfare. More generally, I show that progress towards racial integration would significantly increase the marriage rates among minorities, without reducing the marriage rates of Whites. This finding suggests that the efforts toward making Black women's interracial marriage more attractive and easier to form could lead to better marriage prospects for them, who currently have low marriage rates and a high prevalence of single mothers, as well as for other social groups.

My findings suggest two main avenues for future research. First, it is important to understand the determinants behind varying marital surpluses across interracial marriages. The current estimates only reveal how marital gains differ, but not why. For example, the matching model cannot distinguish why the marital surplus between Black men and White women is higher than the marital surplus between White men and Black women. It would be fruitful to investigate whether these gender differences in marital surplus are affected by economic conditions, residential locations, or other social and cultural factors. Second, the question of which policies can promote interracial marriage needs to be further studied. \cite{Merlinoetal_2019_SchoolPeers} show that greater racial diversity in high school increases interracial dating as adults. It would be fruitful to study whether the policies that promote diversity in other settings would promote interracial marriage and foster social integration.
\pagebreak
\bibliographystyle{chicago}
\bibliography{citation_econ_marriage.bib}



% ================================== APPENDIX ====================================== %
\pagebreak
\appendix


\setcounter{table}{0}
\renewcommand{\thetable}{A\arabic{table}}
\setcounter{figure}{0}
\renewcommand{\thefigure}{A\arabic{figure}}

% ==================== APPENDIX: Individual Gains ===================== %
\section{Appendix: Tables and Figures}

\subsection{Additional Tables}



\begin{table}[H] \caption{Percentage of Other Race and Mixed Races in Each Census Year, Female Aged 35-44, Male Aged 37-46} \label{apptab:otherrace}
\begin{tabular}{lcc}
	\toprule
	Year & Other Race & Mixed Race \\ \midrule 
	1980 & 0.71\% & N/A \\
1990 & 0.78\% & N/A 	\\
2000 & 0.88\% & 2.12\%  \\
2010 & 0.92\% & 1.65\%  \\
2019 & 0.89\% & 2.55\%  \\ \bottomrule 
\end{tabular}
\begin{center}
%\vspace{-7pt}
\begin{minipage}{17cm}
\begin{spacing}{0.9}
{\footnotesize{\underline{Notes:} This table presents the proportion of people who reported Other Race (which includes ``American Indian or Alaska Native" and ``Other race") and Mixed Race among women aged 35-44 and men aged 37-46 for each survey year. A response option of mixed race was added from the 2000 census and onwards. Data sources for this table are: 1960 5\% sample Census, 1970 1\% sample Census, 1980 5\% sample Census, 1990 5\% sample Census, 2000 5\% sample Census, 2010 5\% sample American Community Survey (2006-2010 5 year pooled sample), 2019 5\% sample American Community Survey (2005-2019 5 year pooled sample). Survey weight is applied.}}
\end{spacing}
\end{minipage}
\end{center}
\end{table}

\begin{table}[H] \caption{Percentage of Never Married Singles who Cohabit in Each Census Year, Female Aged 35-44, Male Aged 37-46} \label{apptab:cohabit}
\begin{tabular}{lc}
	\toprule
	Year & \% Cohabiting \\ \midrule 
1990 & 11.3\% 	\\
2000 & 17.7\%  \\
2010 & 21.3\%  \\
2019 & 24.6\%  \\ \bottomrule 
\end{tabular}
\begin{center}
%\vspace{-7pt}
\begin{minipage}{17cm}
\begin{spacing}{0.9}
{\footnotesize{\underline{Notes:} This table presents the proportion of respondents who reported to have cohabiting partners among never-married single women aged 35-44 and never-married single men aged 37-46 for each survey year. A response option for a cohabiting partner was added from the 1990 census and onwards. Data sources for this table are: 1990 5\% sample Census, 2000 5\% sample Census, 2010 5\% sample American Community Survey (2006-2010 5 year pooled sample), 2019 5\% sample American Community Survey (2005-2019 5 year pooled sample). }}
\end{spacing}
\end{minipage}
\end{center}
\end{table}


\subsection{Additional Figures} \label{appsec:figures}



\begin{figure}[H] \caption{Interracial Marriage Rate, Among Married, Age 35-44} \label{fig:intmarriage-allpeople}
	 \includegraphics[scale=0.45]{../Output/CombinedACSCensus_1960-2019/Descriptive/InterracialMarriageRate_includeallind_age35-44.pdf}
\begin{fignote}
	\underline{Note:} This figure shows the proportion of interracial marriage among married men and women aged 35-44 for each survey year. Data sources for this figure are: 1960 5\% sample Census, 1970 1\% sample Census, 1980 5\% sample Census, 1990 5\% sample Census, 2000 5\% sample Census, 2010 5\% sample American Community Survey (2006-2010 5 year pooled sample), 2019 5\% sample American Community Survey (2005-2019 5 year pooled sample). For Hispanics, 1960 and 1970 are excluded as the Hispanic identification is imputed by the IPUMS and does not properly capture the interracial marriage with non-Hispanic whites. Survey weight is applied. 
\end{fignote}

\end{figure}

\pagebreak

\begin{figure}[!htbp] \caption{Interracial/Interethnic Marriage For Each Race/Ethnicity, Among Married, Age 35-44} \label{fig:intmarriage-byrace}
	 \includegraphics[scale=0.5]{../Output/CombinedACSCensus_1960-2019/Descriptive/InterracialMarriage_byrace_1980to2019.pdf}
\begin{fignote}
	\underline{Note:} This figure shows the proportion of those who married out of their race/ethnicity among married men and women aged 35-44 in 1980 and in 2019, respectively. Data sources for this figure are: 1980 5\% sample Census microdata and 2019 5\% sample American Community Survey (2015-2019 5-year pooled sample). Survey weight is applied. 
\end{fignote}
\end{figure}



\begin{figure}[p] \caption{Interracial/Interethnic Marriage Rates Among Hispanics and Asians} \label{fig:intrateha}

 \begin{subfigure}[b]{0.4\textwidth}
           \centering
         \includegraphics[width=\textwidth]{../Output/CombinedACSCensus_1960-2019/Descriptive/InterracialMarriage_amongmarried_MenWomen_r3_clean_age35-44.pdf}
         \caption{Hispanics} \label{fig:outmarriedhisp}
     \end{subfigure}

        \vskip\floatsep
         \begin{subfigure}[b]{0.4\textwidth}
           \centering
         \includegraphics[width=\textwidth]{../Output/CombinedACSCensus_1960-2019/Descriptive/InterracialMarriage_amongmarried_MenWomen_r4_clean_age35-44.pdf}
         \caption{Asian}  \label{fig:outmarriedasian}
     \end{subfigure}
    \begin{fignote}
	\underline{Note:} This figure shows the proportion of those who married out of their race/ethinicity among married individuals of the specified group aged 35-44 in each survey year. ``HSG" refers to high school graduation or the equivalent GED. ``CG" refers to the four-year college degree or above. Data sources for this figure are: 1980 5\% sample Census, 1990 5\% sample Census, 2000 5\% sample Census, 2010 5\% sample American Community Survey (2006-2010 5 year pooled sample), 2019 5\% sample American Community Survey (2005-2019 5 year pooled sample).  Survey weight is applied. 
\end{fignote}
	
\end{figure}



\begin{figure}[H] \caption{Marital Surplus $Z^{IJ}$, 1980} \label{appfig:Z1980}
    \includegraphics[width=\textwidth]{../Output/CombinedACSCensus_1960-2019/Matching/ZIJ_age37_4educrace_year1980.pdf}  
    \begin{fignote} 
\underline{Note:} This figure shows the heatmap for estimated marital surplus $\hat{Z}^{IJ}_t$ for the survey year 1980. Data used to estimate this matrix is described in Section \ref{sec:data}. $I$ refers to husband's type (Row) and $J$ refers to wife's type (Column). 
\end{fignote}  
\end{figure}



\begin{figure}[H] \caption{Marital Surplus $Z^{IJ}$, 2019}  \label{appfig:Z2019}
    \includegraphics[width=\textwidth]{../Output/CombinedACSCensus_1960-2019/Matching/ZIJ_age37_4educrace_year2019.pdf}    
      \begin{fignote} 
\underline{Note:} This figure shows the heatmap for estimated marital surplus $\hat{Z}^{IJ}_t$ for the survey year 2019. Data used to estimate this matrix is described in Section \ref{sec:data}. $I$ refers to husband's type (Row) and $J$ refers to wife's type (Column). 
\end{fignote} 
\end{figure}


\pagebreak


\begin{figure}[H] \caption{Individual Gains from Interracial Marriage}      \label{fig:ipfp2}
 \begin{subfigure}[b]{0.47\textwidth}
           \centering
         \includegraphics[width=\textwidth]{../Output/CombinedACSCensus_1960-2019/Counterfactual/InterPrem_uexpCS_age37_race4educ_hispanic_1960to2019.pdf}
         \caption{Hispanic Men}   \label{fig:ipfp-hispanicmen}
     \end{subfigure}
      \hfill
     \begin{subfigure}[b]{0.47\textwidth}  
          \centering 
        \includegraphics[width=\textwidth]{../Output/CombinedACSCensus_1960-2019/Counterfactual/InterPrem_vexpCS_age37_race4educ_hispanic_1960to2019.pdf}
         \caption{Hispanic Women} \label{fig:ipfp-hispanicwomen}
        \end{subfigure}   
        \vskip\floatsep
         \begin{subfigure}[b]{0.47\textwidth}
           \centering
         \includegraphics[width=\textwidth]{../Output/CombinedACSCensus_1960-2019/Counterfactual/InterPrem_uexpCS_age37_race4educ_asian_1960to2019.pdf}
         \caption{Asian Men} \label{fig:ipfp-asianmen}
     \end{subfigure}
      \hfill
     \begin{subfigure}[b]{0.47\textwidth}  
          \centering 
        \includegraphics[width=\textwidth]{../Output/CombinedACSCensus_1960-2019/Counterfactual/InterPrem_vexpCS_age37_race4educ_asian_1960to2019.pdf}
         \caption{Asian Women} \label{fig:ipfp-asianwomen}
        \end{subfigure}          
       \begin{fignote} 
\underline{Note:} These figures plot the welfare gain from interracial marriage as defined by Equation (\ref{eq:gIm}) and Equation (\ref{eq:gJf}) for each specified type of men and women. Data used to calculate the gains are: 1980-2000 Decennial Census, 2010 and 2019 5-Year ACS. I focus on age 37-46 men and age 35-44 women for each survey year. Further details on the sample restriction are described in Section \ref{sec:data}. Shade for each line refers to the 95\% confidence interval. Standard errors are calculated from the sampling variation in the data. $HSD$: high school dropout, $HSG$: high school graduate with no college education, $SC$: less than 4 years of college education, and $CG$: 4 years of college education or more.
\end{fignote}
\end{figure}



\begin{figure}[H]  \caption{Percentage Point Changes in Marriage Rates after Removing Gender Gap in $\mathbf{Z}^{interracial}$ for Black People, Year 2019} \label{appfig:cfgendergap-nonBlack}
  \begin{subfigure}[b]{0.47\textwidth}
           \centering
         \includegraphics[width=\textwidth]{../Output/CombinedACSCensus_1960-2019/Counterfactual/Counterfactual_gendergap_Whitewomen_37_4educrace_year2019.pdf}
         \caption{White Women} \label{fig:cf1-Whitewomen}
     \end{subfigure}
      \hfill
     \begin{subfigure}[b]{0.47\textwidth}  
          \centering 
        \includegraphics[width=\textwidth]{../Output/CombinedACSCensus_1960-2019/Counterfactual/Counterfactual_gendergap_Whitemen_37_4educrace_year2019.pdf}
         \caption{White Men} \label{fig:cf1-Whitemen}
        \end{subfigure}
        \vskip\floatsep
          \begin{subfigure}[b]{0.47\textwidth}
           \centering
          \includegraphics[width=\textwidth]{../Output/CombinedACSCensus_1960-2019/Counterfactual/Counterfactual_gendergap_Hispanicwomen_37_4educrace_year2019.pdf}
         \caption{Hispanic Women} \label{fig:cf1-Hispanicwomen}
     \end{subfigure}
      \hfill
     \begin{subfigure}[b]{0.47\textwidth}  
          \centering 
        \includegraphics[width=\textwidth]{../Output/CombinedACSCensus_1960-2019/Counterfactual/Counterfactual_gendergap_Hispanicmen_37_4educrace_year2019.pdf}
         \caption{Hispanic Men} \label{fig:cf1-Hispanicwomen}
        \end{subfigure}
        
         \begin{fignote}
	\underline{Note:} This figure plots the percentage point differences in marriage rate between actual marriage rates and counterfactual marriage rates for each group: (Counterfactual marriage rate - Actual marriage rate). Counterfactuals consider two different scenarios of removing the gender gap in the marital surplus from interracial marriage for Black people. As described in \ref{sec:cfgendergap}, \textbf{Counterfactual 1} sets the marital surpluses of \textit{Black women}'s interracial marriages the same as those of \textit{Black men}'s interracial marriages (Equation \ref{eq:cf1gendergap}). Alternatively,  \textbf{Counterfactual 2} sets the marital surpluses of \textit{Black men}'s interracial marriages the same as those of \textit{Black women}'s interracial marriages (Equation \ref{eq:cf2gendergap}). 
\end{fignote}   
\end{figure}

\begin{figure}[H]  \caption{Percentage Point Changes in Marriage Rates after Removing Gender Gap in $\mathbf{Z}^{interracial}$ for Black People, Year 2019 (Continued)} \label{appfig:cfgendergap-nonBlack2}
  \begin{subfigure}[b]{0.47\textwidth}
           \centering
         \includegraphics[width=\textwidth]{../Output/CombinedACSCensus_1960-2019/Counterfactual/Counterfactual_gendergap_Asianwomen_37_4educrace_year2019.pdf}
         \caption{Asian Women} \label{fig:cf1-Asianwomen}
     \end{subfigure}
      \hfill
     \begin{subfigure}[b]{0.47\textwidth}  
          \centering 
        \includegraphics[width=\textwidth]{../Output/CombinedACSCensus_1960-2019/Counterfactual/Counterfactual_gendergap_Asianmen_37_4educrace_year2019.pdf}
         \caption{Asian Men} \label{fig:cf1-Asianmen}
        \end{subfigure}
        
         \begin{fignote}
	\underline{Note:} This figure plots the percentage point differences in marriage rate between actual marriage rates and counterfactual marriage rates for each group: (Counterfactual marriage rate - Actual marriage rate). Counterfactuals consider two different scenarios of removing the gender gap in the marital surplus from interracial marriage for Black people. As described in \ref{sec:cfgendergap}, \textbf{Counterfactual 1} sets the marital surpluses of \textit{Black women}'s interracial marriages the same as those of \textit{Black men}'s interracial marriages (Equation \ref{eq:cf1gendergap}). Alternatively,  \textbf{Counterfactual 2} sets the marital surpluses of \textit{Black men}'s interracial marriages the same as those of \textit{Black women}'s interracial marriages (Equation \ref{eq:cf2gendergap}). 
\end{fignote}   
\end{figure}


 \begin{figure}[H] \caption{Marital Surplus Matrix in 2019, Actual vs. Complete Integration}      \label{fig:Zcompare}
 %\ContinuedFloat
  %  \captionsetup{list=off,format=cont}
    \begin{subfigure}[b]{0.47\textwidth}  
          \centering 
        \includegraphics[width=\textwidth]{../Output/CombinedACSCensus_1960-2019/Counterfactual/ZIJ_compareint_age37_4educrace_year2019_notext.pdf}
         \caption{Actual $\hat{\mathbf{Z}}_{2019}$}  
        \end{subfigure}
      \hfill
  \begin{subfigure}[b]{0.47\textwidth}
           \centering
         \includegraphics[width=\textwidth]{../Output/CombinedACSCensus_1960-2019/Counterfactual/Z_integration1_age37_4educrace_year2019_notext.pdf}
         \caption{Counterfactual $\hat{\mathbf{Z}}^{Integrated}_{2019}$} 
     \end{subfigure}
  
        \begin{fignote} 
\underline{Note:} This figure shows heatmap of the marital surplus $Z^{IJ}_{2019}$ for the actual values that is estimated from data (Panel (a)) and the counterfactual values with complete integration (Panel (b)) and respectively. $I$ refers to husband's type (Row) and $J$ refers to wife's type (Column). 
\end{fignote}
\end{figure}






\begin{figure}[H] \caption{Simulated Rate of Singlehood for Varying Degree of Racial/Ethnic Integration} \label{fig:integrationVarying2}

 \begin{subfigure}[b]{0.47\textwidth}
           \centering
         \includegraphics[width=\textwidth]{../Output/CombinedACSCensus_1960-2019/Counterfactual/CompleteIntegration_VaryingPercent_Hispanicmen_2019.pdf}
         \caption{Hispanic Men} \label{fig:integrationHispanicmen}
     \end{subfigure}
	  \hfill
	   \begin{subfigure}[b]{0.47\textwidth}
           \centering
         \includegraphics[width=\textwidth]{../Output/CombinedACSCensus_1960-2019/Counterfactual/CompleteIntegration_VaryingPercent_Hispanicwomen_2019.pdf}
         \caption{Hispanic Women} \label{fig:integrationHispanicwomen}
     \end{subfigure}
        \vskip\floatsep
         \begin{subfigure}[b]{0.47\textwidth}
           \centering
         \includegraphics[width=\textwidth]{../Output/CombinedACSCensus_1960-2019/Counterfactual/CompleteIntegration_VaryingPercent_Asianmen_2019.pdf}
         \caption{Asian Men}  \label{fig:integrationAsianmen}
     \end{subfigure}
     \hfill 
         \begin{subfigure}[b]{0.47\textwidth}
           \centering
         \includegraphics[width=\textwidth]{../Output/CombinedACSCensus_1960-2019/Counterfactual/CompleteIntegration_VaryingPercent_Asianwomen_2019.pdf}
         \caption{Asian Women}  \label{fig:integrationAsianwomen}
     \end{subfigure}
    \begin{fignote}
	\underline{Note:} This figure plots the simulated rate of singlehood at each \% of ethnic/racial integration (rescaled $p$) for the specified group. ``\% Racially Integrated" describes how close the counterfactual marital surplus is to the complete integration case. Estimation is done using the 2019 Census data. I focus on age 37-46 men and age 35-44 women. Further details on the sample restriction are described in Section \ref{sec:data}.
\end{fignote}
	
\end{figure}





\subsection{Sensitivity Check: Excluding Cohabiting Singles} \label{appsec:excludecohabit}

As shown in Table \ref{apptab:cohabit}, the proportion of never-married singles who cohabit with a partner has increased over time. To see how the cohabiting singles affect the results, I perform sensitivity analyses that exclude cohabiting singles from the single population. I re-estimate the welfare gains from marital desegregation for each group, which is presented in Figure \ref{fig:ipfp-excl}. The results confirm that excluding cohabiting singles do not affect the results for welfare gain from marital desegregation. 

\begin{figure}[p] \caption{Individual Gains from Interracial Marriage, Excluding Cohabiting Singles}      \label{fig:ipfp-excl}
 \begin{subfigure}[b]{0.47\textwidth}
           \centering
         \includegraphics[width=\textwidth]{../Output/CombinedACSCensus_1960-2019/Counterfactual/InterPrem_uexpCS_age37_race4educ_black_1960to2019_excludecohabit.pdf}
         \caption{Black Men}   \label{fig:ipfp-blackmen-excl}
     \end{subfigure}
      \hfill
     \begin{subfigure}[b]{0.47\textwidth}  
          \centering 
        \includegraphics[width=\textwidth]{../Output/CombinedACSCensus_1960-2019/Counterfactual/InterPrem_vexpCS_age37_race4educ_black_1960to2019_excludecohabit.pdf}
         \caption{Black Women} \label{fig:ipfp-blackwomen-excl}
        \end{subfigure}   
        \vskip\floatsep
         \begin{subfigure}[b]{0.47\textwidth}
           \centering
         \includegraphics[width=\textwidth]{../Output/CombinedACSCensus_1960-2019/Counterfactual/InterPrem_uexpCS_age37_race4educ_white_1960to2019_excludecohabit.pdf}
         \caption{White Men} \label{fig:ipfp-whitemen-excl}
     \end{subfigure}
      \hfill
     \begin{subfigure}[b]{0.47\textwidth}  
          \centering 
        \includegraphics[width=\textwidth]{../Output/CombinedACSCensus_1960-2019/Counterfactual/InterPrem_vexpCS_age37_race4educ_white_1960to2019_excludecohabit.pdf}
         \caption{White Women} \label{fig:ipfp-whitewomen-excl}
        \end{subfigure}          
       \begin{fignote} 
\underline{Note:} These figures plot the individual gains from interracial marriage as defined by Equation (\ref{eq:gIm}) and Equation (\ref{eq:gJf}) for each specified type of men and women. Data used to calculate the gains are: 1980-2000 Decennial Census, 2010 and 2019 5-Year ACS. I focus on age 37-46 men and age 35-44 women for each survey year. Further details on the sample restriction are described in Section \ref{sec:data}. I exclude cohabiting singles from the estimation sample. Shade for each line refers to the 95\% confidence interval. Standard errors are calculated from the sampling variation in the data. 
\end{fignote}
\end{figure}





% ==================== APPENDIX: DECOMPOSITION ===================== %

\begin{landscape}
\section{Appendix: Decomposition} \label{appsec:decomp}

\subsection{Full Solution of IFT Partials}  \label{appsec:decomp-method}

\textbf{Full solution for the IFT partials:} Full solution for the Jacobian matrix (Equation \ref{eq:IFT}) is as follows:

\begin{align*}
\begin{Large}
	\begin{bmatrix}
		 \frac{\partial \mathbf{s} }{\partial \tilde{\boldsymbol{\theta}}}
	\end{bmatrix}_{(2K) \times (2K+K^2)}
	= - 
	\underbrace{\begin{bmatrix}
		\frac{\partial \mathbf{F}}{\partial  \mathbf{s} } \\ \addlinespace 
		\frac{\partial \mathbf{G}}{\partial  \mathbf{s} } 
	\end{bmatrix}^{-1}_{(2K)\times (2K)}}_{\color{blue} [A]}
	\underbrace{\begin{bmatrix}
		\frac{\partial \mathbf{F}}{\partial \tilde{\boldsymbol{\theta}} } \\ \addlinespace 
		\frac{\partial \mathbf{G}}{\partial \tilde{\boldsymbol{\theta}}}
	\end{bmatrix}_{(2K) \times (2K+K^2)}}_{\color{blue} [B]}
	\end{Large} 
\end{align*}
where 
\begin{align*}
	{\color{blue} [A]} = 
	\begin{bmatrix}
	2s_{1\emptyset} + \sum_J \tilde{Z}_{1J}s_{\emptyset J} & 0 & \cdots & 0 & \tilde{Z}_{11}s_{1\emptyset} & \tilde{Z}_{12}s_{1\emptyset} & \cdots & \tilde{Z}_{1K}s_{1\emptyset} \\
	0 & 2s_{2\emptyset} + \sum_J \tilde{Z}_{2J}s_{\emptyset J} & \cdots & 0 & \tilde{Z}_{21}s_{2\emptyset} & \tilde{Z}_{22}s_{2\emptyset} & \cdots & \tilde{Z}_{2K}s_{2\emptyset} \\
	\vdots & \vdots & \ddots & \vdots &  \vdots & \vdots &  & \vdots \\
	0 & 0 & \cdots & 2s_{K\emptyset} + \sum_J \tilde{Z}_{KJ}s_{\emptyset J} & \tilde{Z}_{K1}s_{K\emptyset} & \tilde{Z}_{K2}s_{K\emptyset} & \cdots & \tilde{Z}_{KK}s_{K\emptyset} \\
	\tilde{Z}_{11}s_{\emptyset1} & \tilde{Z}_{21}s_{\emptyset1} & \cdots & \tilde{Z}_{K1}s_{\emptyset1} & 2s_{\emptyset 1} + \sum_I \tilde{Z}_{I1}s_{I \emptyset } & 0 & \cdots & 0 \\
	\tilde{Z}_{12}s_{\emptyset2} & \tilde{Z}_{22}s_{\emptyset2} & \cdots & \tilde{Z}_{K2}s_{\emptyset2} & 0 & 2s_{\emptyset 2} + \sum_I \tilde{Z}_{I2}s_{I \emptyset } & \cdots & 0 \\
	\vdots & \vdots &  & \vdots & \vdots & \vdots & \ddots & \vdots  \\
	\tilde{Z}_{1K}s_{\emptyset K} & \tilde{Z}_{2K}s_{\emptyset K} & \cdots & \tilde{Z}_{KK}s_{\emptyset K} & 0 & 0 & \cdots & 2s_{\emptyset K} + \sum_I \tilde{Z}_{IK}s_{I \emptyset }
	\end{bmatrix}^{-1}
\end{align*}
and
\setcounter{MaxMatrixCols}{25}
\begin{align*}
	{\color{blue} [B]} = 
	\begin{bmatrix}
		-1 & 0 & \cdots & 0 & 0 & 0 & \cdots & 0 & s_{1\emptyset}s_{\emptyset 1} & s_{1\emptyset}s_{\emptyset 2} & \cdots & s_{1\emptyset}s_{\emptyset K} & 0 & 0 & \cdots & 0 & \cdots & 0 & 0 & \cdots & 0 \\
		0 & -1 & \cdots & 0 & 0 & 0 & \cdots & 0  & 0 & 0 & \cdots & 0 & s_{2\emptyset}s_{\emptyset 1} & s_{2\emptyset}s_{\emptyset 2} & \cdots & s_{2\emptyset}s_{\emptyset K} & \cdots & 0 & 0 & \cdots & 0  \\
		\vdots & \vdots & \ddots & \vdots & \vdots & \vdots & & \vdots & \vdots & \vdots & & \vdots & \vdots & \vdots  & & \vdots & & \vdots & \vdots & & \vdots \\ 
		0 & 0 & \cdots & -1 & 0 & 0 & \cdots & 0  & 0 & 0 & \cdots & 0  & \cdots & 0 & 0 & 0 & \cdots & s_{K\emptyset}s_{\emptyset 1} & s_{K\emptyset}s_{\emptyset 2} & \cdots & s_{K\emptyset}s_{\emptyset K} \\
		0 & 0 & \cdots & 0 & -1 & 0 & \cdots & 0 & s_{1\emptyset}s_{\emptyset 1} & 0 & \cdots & 0 & s_{2\emptyset}s_{\emptyset 1} & 0 & \cdots & 0 & \cdots & s_{K\emptyset}s_{\emptyset 1} & 0 & \cdots & 0 \\
		0 & 0 & \cdots & 0 & 0 & -1 & \cdots & 0 & 0 & s_{1\emptyset}s_{\emptyset 2}  & \cdots & 0 & 0 & s_{2\emptyset}s_{\emptyset 2} & \cdots & 0 & \cdots & 0 & s_{K\emptyset}s_{\emptyset 2}& \cdots & 0 \\ 
		\vdots & \vdots & & \vdots & \vdots & \vdots & \ddots & \vdots & \vdots & \vdots & \ddots & \vdots & \vdots & \vdots & \ddots & \vdots & & \vdots & \vdots & \ddots & \vdots  \\
		0 & 0 & \cdots & 0 & 0 & 0 & \cdots & -1 & 0 & 0  & \cdots & s_{1\emptyset}s_{\emptyset K} & 0 & 0 & \cdots & s_{2\emptyset}s_{\emptyset K} & \cdots & 0 & 0 & \cdots & s_{K\emptyset}s_{\emptyset K}
	\end{bmatrix}
\end{align*}

\vspace{5mm}
Estimation of the Jacobian matrix is done by combining  {\color{blue} [A]} and {\color{blue} [B]} using matrix multiplication.  

\end{landscape}


\subsection{Details on Decomposition Procedures} \label{appsec:decompdetails}

In this section, I describe the estimation steps to decompose the expected utility of type $I$ men. The application to the individual welfare gain from access to interracial marriage, which is a function of expected utilities, can be done analogously. 

\vspace{3mm}
\noindent\textbf{STEP 1:} First, to link the change in the expected utility to the IFT partials, I take the total differential of the expected utility:
\begin{align}
	d\bar{u}^I &= \frac{1}{n^I}dn^I -  \frac{2}{s^{I\emptyset}} \Big(  \underbrace{\frac{\partial s^{I\emptyset} }{\partial \tilde{\boldsymbol{\theta}}}}_{From \; IFT} d \tilde{\boldsymbol{\theta}}  \Big) \label{appeq:duI}
\end{align}


\noindent\textbf{STEP 2:} A naive way of expressing the changes in $\bar{u}^I$ from year 1980 to 2019 using Equation (\ref{eq:duI}) is the following:
\begin{align*}
	\Delta^{2019-1980} \bar{u}^I = \frac{1}{n^I} \Delta^{2019-1980} n^I -  \frac{2}{s^{I\emptyset}} \Big(  \frac{\partial s^{I\emptyset} }{\partial \tilde{\boldsymbol{\theta}}} \Delta^{2019-1980} \tilde{\boldsymbol{\theta}}  \Big)
\end{align*}
where $\Delta^{2019-1980} y$ refers to change in $y$ from 1980 to 2019. However, this is problematic because the implicit function theorem and the total differentials only give good approximations for \textit{very small} changes in the model primitives. US has experienced large changes in population distribution over the past four decades. Moreover, marital surplus $\mathbf{Z}$ also has experienced changes over time. Hence, it is improper to use 40 years of changes to evaluate Equation (\ref{appeq:duI}). 

A better, but still not ideal, approach is to divide the time period into smaller time periods based on available survey years. Because I use the census data with 10-year intervals, $\Delta^{2019-1980} \bar{u}^I_t$ can be decomposed into:
\begin{align*}
	\Delta^{2019-1980} \bar{u}^I = \Delta^{1990-1980} \bar{u}^I +  \Delta^{2000-1990} \bar{u}^I + \Delta^{2010-2000} \bar{u}^I + \Delta^{2019-2010} \bar{u}^I 
\end{align*}
%where 
%\begin{align}
%	\Delta^{t_2-t_1} \bar{u}^I = \frac{1}{n^I_{t_1}} \Delta^{t_2-t_1} n^I -  \frac{2}{s^{I\emptyset}_{t_1}} \Big( \Big[ \frac{\partial s^{I\emptyset} }{\partial \boldsymbol{\tilde{\theta}}}  \Big]_{t_1} \Delta^{t_2-t_1} \boldsymbol{\tilde{\theta}}  \Big) \label{eq:deltat}
%\end{align}
However, changes in model primitives over each decade may still be considered large. 

In order to better approximate the effect of changes in model primitives on $d\bar{u}^I_t$, I implement the homotopy method following \cite{Judd_1998_book}. This method decomposes the large changes in the model primitives into a series of infinitesimal changes. I apply this method for each decade based on the available survey years: 1980 to 1990, 1990 to 2000, 2000 to 2010, and 2010 to 2019. 

\vspace{4mm}
To give a concrete example, I consider the changes from 1980 to 1990. Let me denote 1980 as $\tau = 0$ and 1990 as $\tau = 1$. Then $\tilde{\boldsymbol{\theta}}_0$ (resp. $\tilde{\boldsymbol{\theta}}_1$)  is the vector of the values of model primitives in 1980 (resp. in 1990). Then I consider the homotopy:
\begin{align*}
	\tilde{\boldsymbol{\theta}}_\tau = \tau \tilde{\boldsymbol{\theta}}_1 + (1-\tau) \tilde{\boldsymbol{\theta}}_0, \quad \tau \in [0,1]
\end{align*}
which defines a series of intermediate values of the model primitives with interval $d\tau$ between observed values at $\tau = 0$ and $\tau = 1$. Because $\tilde{\boldsymbol{\theta}}_\tau$ is now a function of $\tau$ as defined above, $d\tilde{\boldsymbol{\theta}}_\tau$ becomes $d\tilde{\boldsymbol{\theta}}_\tau = (\tilde{\boldsymbol{\theta}}_1 - \tilde{\boldsymbol{\theta}}_0)d\tau$. Then, applying the homotopy to Equation (\ref{appeq:duI}),
\begin{align}
	d\bar{u}^I &= \frac{1}{n^I_\tau}(n^I_1 - n^I_0)d\tau -  \frac{2}{s^{I\emptyset}} \Big(  \Big[\frac{\partial s^{I\emptyset} }{\partial \tilde{\boldsymbol{\theta}}_\tau} \Big]_\tau (\tilde{\boldsymbol{\theta}}_1 - \tilde{\boldsymbol{\theta}}_0)d\tau  \Big) \label{appeq:duI-homotopy}
\end{align}
where $\Big[\frac{\partial s^{I\emptyset}}{\partial \tilde{\boldsymbol{\theta}}_t} \Big]_\tau$ means that this partial is evaluated at each $\tau$. Note that $s^{I\emptyset}_\tau$ is updated as $\tau$ progresses with interval $d\tau$. 

I use Equation (\ref{appeq:duI-homotopy}) to estimate $d\bar{u}^I$ for each decade and to decompose $d\bar{u}^I$ into contributions by change in each of the model primitives. With the homotopy method, I can use infinitesimal change $dt$ to evaluate and decompose $\Delta^{(\tau+d\tau)-\tau} \bar{u}^I$ for $\tau \in [0,1]$. I specify $d\tau = 0.001$ when estimating Equation (\ref{appeq:duI-homotopy}) for each decade. Summing $\Delta^{(\tau+d\tau)-\tau}\bar{u}^I$ over all  $\tau \in [0,1]$ gives better approximation of $\Delta \bar{u}^I$ than using the observed 10-year changes of model primitives to evaluate Equation (\ref{appeq:duI}). 

For a more concrete illustration, I describe in detail how I perform first few steps for this fine-tuning method: 
\begin{itemize}
	\item \textbf{STEP 2.1:} From $\tau = 0 \rightarrow \tau = 0.001$
	
	The goal is to estimate $\bar{u}^I_{0.001}$. Starting from $\bar{u}^I_0$, 
	\begin{align*}
		\bar{u}^I_{0.001} = \bar{u}^I_0 + d\bar{u}^I_0
	\end{align*}
	Using the fine-tuning method, $d\bar{u}^I_0$ is expressed as:
	\begin{align*}
		d\bar{u}^I_0 = \frac{1}{n^I_0} (n^I_1 - n^I_0)\cdot 0.001 - \frac{2}{s^{I\emptyset}_0} \frac{\partial s^{I\emptyset}_0}{\partial \tilde{\boldsymbol{\theta}}_\tau} (\tilde{\boldsymbol{\theta}}_1 - \tilde{\boldsymbol{\theta}}_0)\cdot 0.001
	\end{align*}
	Note that $\frac{\partial s^{I\emptyset}_0}{\partial \tilde{\boldsymbol{\theta}}_\tau}$ is a function of $s^{I\emptyset}_0$, $s^{\emptyset J}_0$, $Z^{IJ}_0$, all of which are evaluated at $\tau = 0$. 
	
	In this step, I also need to compute $s^{I\emptyset}_{0.001}$ and $s^{\emptyset J}_{0.001}$, because these will be used in the next step. For example, 
	\begin{align*}
		s^{I\emptyset}_{0.001} &= s^{I\emptyset}_{0} + d s^{I\emptyset}_{0} \\
		&= s^{I\emptyset}_{0} + \frac{\partial s^{I\emptyset}_0}{\partial \tilde{\boldsymbol{\theta}}_\tau} (\tilde{\boldsymbol{\theta}}_1 - \tilde{\boldsymbol{\theta}}_0)\cdot 0.001
	\end{align*}
\end{itemize}
\begin{itemize}	
	\item \textbf{STEP 2.2:} From $\tau = 0.001 \rightarrow \tau = 0.002$.
	
	
	The goal is to estimate $\bar{u}^I_{0.002}$. Starting from $\bar{u}^I_{0.001}$, 
	\begin{align*}
		\bar{u}^I_{0.002} = \bar{u}^I_{0.001} + d\bar{u}^I_{0.001}
	\end{align*}
	Using the fine-tuning method, $d\bar{u}^I_{0.001}$ is expressed as:
	\begin{align*}
		d\bar{u}^I_{0.001} = \frac{1}{n^I_{0.001}} (n^I_1 - n^I_0)\cdot 0.001 - \frac{2}{s^{I\emptyset}_{0.001}} \frac{\partial s^{I\emptyset}_{0.001}}{\partial \tilde{\boldsymbol{\theta}}_\tau} (\tilde{\boldsymbol{\theta}}_1 - \tilde{\boldsymbol{\theta}}_0)\cdot 0.001
	\end{align*}
	where $n^I_{0.001} = 0.001 n^I_1 + 0.999 n^I_0$.
	
	Note that $\frac{\partial s^{I\emptyset}_{0.001}}{\partial \tilde{\boldsymbol{\theta}}_\tau}$ is a function of $s^{I\emptyset}_{0.001}$, $s^{\emptyset J}_{0.001}$, and $Z^{IJ}_{0.001}$. I have already estimated $s^{I\emptyset}_{0.001}$ and $s^{\emptyset J}_{0.001}$ from the previous step, and $Z^{IJ}_{0.001} = 0.001Z^{IJ}_1 + 0.999 Z^{IJ}_0$. 
	
	
	In this step, I also need to compute $s^{I\emptyset}_{0.002}$ and $s^{\emptyset J}_{0.002}$, because these will be used in the next step. For example, 
	\begin{align*}
		s^{I\emptyset}_{0.002} &= s^{I\emptyset}_{0.001} + d s^{I\emptyset}_{0.001} \\
		&= s^{I\emptyset}_{0.001} + \frac{\partial s^{I\emptyset}_{0.001}}{\partial \tilde{\boldsymbol{\theta}}_\tau} (\tilde{\boldsymbol{\theta}}_1 - \tilde{\boldsymbol{\theta}}_{0})\cdot 0.001
	\end{align*}
	
	\item \textbf{STEP 2.3 and above:} The rest of the estimation proceeds analogously until $\tau$ reaches 1. 
\end{itemize}


\vspace{5mm} 
\noindent\textbf{STEP 3:} I now explain how to decompose the changes from 1980 to 2019 in individual expected utilities $\bar{u}^I$ into contributions by each model primitive. As an example, let's consider how $\Delta^{1990-1980} \bar{u}^I$ is estimated according to Equation (\ref{appeq:duI-homotopy}):
\begin{align*}
	\Delta^{1990-1980} \bar{u}^I &= \sum_{\tau \in [0,1], d\tau = 0.001} \frac{1}{n^I_\tau}(n^I_1 - n^I_0)d\tau -  \frac{2}{s^{I\emptyset}} \Big(  \Big[\frac{\partial s^{I\emptyset} }{\partial \tilde{\boldsymbol{\theta}}_\tau} \Big]_\tau (\tilde{\boldsymbol{\theta}}_1 - \tilde{\boldsymbol{\theta}}_0)d\tau  \Big) 
\end{align*}
where $\tau = 0$ refers to year 1980 and $\tau = 1$ refers to year 1990. 

Because $\Delta^{1990-1980}\bar{u}^I$ is a linear function in $(\tilde{\theta}_1 - \tilde{\theta}_0)$, it can be linearly decomposed into parts that are attributed to each model primitive $\theta^k$.\footnote{For example, the part of $\Delta^{1990-1980}\bar{u}^I$ that is contributed by the number of $WhiteHSG$ women is $\sum_{\tau \in [0,1], d\tau = 0.001} -  \frac{2}{s^{I\emptyset}} \Big(  \Big[\frac{\partial s^{I\emptyset} }{\partial m^{WhiteHSG}_\tau} \Big]_\tau (m^{WhiteHSG}_1 - m^{WhiteHSG}_0)d\tau  \Big)$.} I call this the \textbf{contribution} of $\theta^k$ to $\Delta^{1990-1980}\bar{u}^I$. The contribution of $\theta^k$ is essentially the change in $\theta^k$ from 1980 to 1990 multiplied by a multiplier that measures how sensitive $\bar{u}^I$ is with respect to the change in $\theta^k$. Because summing up all contributions of the model primitives leads to $\Delta^{1990-1980}\bar{u}^I$, each contribution can be thought of as a portion of the changes in the expected utilities that is attributed to $\theta^k$. In order to decompose changes in $\bar{u}^I$ over a longer time frame from 1980 to 2019, I simply sum up all four decade-by-decade contributions of each model primitive.

\vspace{2mm}

While I only described the decomposition steps for $\bar{u}^I$ for the illustration purpose, the decomposition for the welfare gains, which is $\bar{u}^{I, actual} - \bar{u}^{I, counterfactual}$, is straightforward. 


\subsection{Additional Decomposition Results }

\begin{table}[H] \caption{Decomposition: Top three contribution from changes in $\mathbf{Z}$, Black College Graduates} \label{tab:blackHSG-decomp-top5}
\centering
\scalebox{1}{
\begin{tabular}{llccc}
 \toprule

   \multicolumn{5}{l}{A. Decomposition for BlackHSG Men's $\Delta$ Welfare Gain \quad ($\Delta Gain = 1.3$) } \\[0.3cm]
 
\quad Contribution & Top (+) & 1.1 & 0.7 & 0.6 \\ 
   &  & $\downarrow$ $Z^{WhiteHSG, WhiteHSG}$ & $\uparrow$ $Z^{BlackHSG,WhiteCG}$ & $\downarrow$ $Z^{WhiteSC,WhiteHSG}$  \\ 
   & Top (-) & -0.6 & -0.4 & -0.4  \\ 
   &  & $\downarrow$ $Z^{BlackHSG,AsianHSD}$ & $\downarrow$ $Z^{BlackHSG,WhiteHSG}$ & $\downarrow$ $Z^{BlackHSG, HispHSD}$  \\  \midrule 
     \multicolumn{4}{l}{B. Decomposition for BlackHSG Women's $\Delta$ Welfare Gain \quad  ($\Delta Gain = 2.6$)} \\[0.3cm]
    \quad Contribution & Top (+) &1.5 & 1.2 & 0.7  \\ 
   &  &  $\downarrow$ $Z^{BlackHSG,BlackHSG}$ & $\downarrow$ $Z^{BlackSC,BlackHSG}$ & $\downarrow$ $Z^{BlackHSD, BlackHSG}$ \\ 
   & Top (-) & -0.2 & -0.2 & -0.2 \\ 
   &  & $\downarrow$ $Z^{WhiteCG, WhiteSC}$ &  $\downarrow$ $Z^{BlackHSG, BlackHSD}$ & $\downarrow$ $Z^{BlackSC,BlackSC}$\\ 
   \bottomrule
\end{tabular}}
\begin{center}
%\vspace{-7pt}
\begin{minipage}{17cm}
\begin{spacing}{0.8}
{\footnotesize{\underline{Notes:} This table presents the top three positive and negative contributions from marital surplus to the 1980-2019 changes in the welfare gains from interracial marriage for Black high school graduate men (Panel A) and Black high school graduate women (Panel B). For marital surplus $Z^{IJ}$, $I$ refers to husband's type and $J$ refers to wife's type. Upward arrow ($\uparrow$) indicates that the corresponding marital surplus has increased over the analysis period, and downward arrow  ($\downarrow$)  indicates that it has decreased over the analysis period. }}
\end{spacing}
\vspace{-3ex}
\end{minipage}
\end{center}
\end{table}

\begin{figure}[H] \caption{Details on the contributions from the changes in non-White population} \label{fig:decomp-whiteCG-pop-details2}

 \begin{subfigure}[b]{0.9\textwidth}
           \centering
         \includegraphics[width=\textwidth]{../Output/CombinedACSCensus_1960-2019/Counterfactual/PartialDiff_fempop_race4educ_WhiteCGmen_1960to2019.pdf}
         \caption{For WhiteCG Men's Welfare Gains: Role of Female Population Changes}
     \end{subfigure}
	  \hfill
	   \begin{subfigure}[b]{0.9\textwidth}
           \centering
         \includegraphics[width=\textwidth]{../Output/CombinedACSCensus_1960-2019/Counterfactual/PartialDiff_malepop_race4educ_WhiteCGmen_1960to2019.pdf}
         \caption{For WhiteCG Men's Welfare Gains: Role of Male Population Changes}
     \end{subfigure}
     \vskip\floatsep
      \begin{subfigure}[b]{0.9\textwidth}
           \centering
         \includegraphics[width=\textwidth]{../Output/CombinedACSCensus_1960-2019/Counterfactual/PartialDiff_fempop_race4educ_WhiteCGwomen_1960to2019.pdf}
         \caption{For WhiteCG Women's Welfare Gains: Role of Female Population Changes}
     \end{subfigure}
	  \hfill
	   \begin{subfigure}[b]{0.9\textwidth}
           \centering
         \includegraphics[width=\textwidth]{../Output/CombinedACSCensus_1960-2019/Counterfactual/PartialDiff_malepop_race4educ_WhiteCGwomen_1960to2019.pdf}
         \caption{For WhiteCG Women's Welfare Gains: Role of Male Population Changes}
     \end{subfigure}
    \begin{fignote}
	\underline{Note:} This figure presents the decomposition of the 1980-2019 changes in the welfare gains from interracial marriage for WhiteCG men and WhiteCG women. This figure focuses on the contributions of the changes in different-race populations. Each column shows the contribution made by the change in the corresponding population. 
\end{fignote}
	
\end{figure}

\end{document}